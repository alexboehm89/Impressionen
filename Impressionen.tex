\documentclass[11pt]{book}
\usepackage[utf8]{inputenc}
\title{Impressionen}
\author{Alexander Böhm}
\begin{document}
\maketitle
\chapter{Vorbereitungen}
1. September 2019, 12:04 im Zug von München nach Paris, es ist grau und regnet leicht

Wir sitzen im Zug. Nach sehr anstrengenden letzten Tagen haben wir es geschafft mit voll gepacktem
Rucksack aufzubrechen. Kurz vor der Abreise wurde es, wie immer, stressig. Es gibt einfach zu vieles
Dinge die man zu erledigen hat, Wohnung leer räumen, sich von Freunden verabschieden, Arzttermine, 
Behördengänge, und bei all dem nicht den Kopf verlieren während man panisch versucht nichts zu 
vergessen. Ich hatte erwartet, dass sich die Erleichterung, das Gefühl der Entspannung, im Zug 
einstellen würden. Aber ich habe ab dem Zeitpunkt ab dem ich die Wohnung verlassen habe auch den 
Stress hinter mir gelassen. Im Moment empfinde ich nur sehr große Vorfreude, eine schöne Mischung
aus positiver Nervosität und Aufregung. Dies führt dazu, dass ich trotz der körperlichen Müdigkeit
nicht in der Lage zu schlafen. Blandine und ich sind beide froh, dass wir noch 2 Tage in
Paris verbringen können bevor wir wirklich nach Hongkong aufbrechen. Ich denke, dass so ein verzögerter Start doch 
einige Ruhe mit sich bringt, da man nicht direkt zum Flughafen hetzen muss. 

3. September 2018, 14:45 am Gate, die Sonne scheint

Wir sitzen am Gate C91 und der A380-800 wartet schon auf uns. Das Gepäck ist abgegeben, zwei gut 
gepackte Rucksäcke die vom Gewicht etwas ungleich verteilt sind. Meiner wiegt 17 kg während 
Blandines bei ungefähr 11kg eingecheckt wurde. Das wird im Laufe der Reise sicherlich noch etwas 
besser aufgeteilt werden, aber mein Rucksack wirkt jetzt schon fast leer. Wir haben wohl zu 
effizient gepackt. Blandine wirkt sehr entspannt und ruhig, auch wenn sie innerlich sehr aufgeregt 
ist. Ihre Vorfreude auf diese Reise ist ihr die ganze Zeit anzumerken. Ich freue mich so sehr mit 
ihr diese Erfahrung zu machen. 

\chapter{China}

4 September 2018, 21:30 im Hostelzimmer, es ist warm draußen aber die Luft ist deutlich abgekühlt

Der eigentliche Beginn der Reise ist für mich heute, als ich das erste Mal ungefilterte Luft in 
Hongkong eingeatmet habe. Es war unerwartet heiß und stickig. Blandine und ich sind nach halbstündiger
Fahrt vom Flughafen an der Kowloon-Station ausgestiegen. Nachdem wir uns mit der Hilfe von drei 
Rolltreppen an die Oberfläche gekämpft haben, waren wir endlich in Hongkong. Es war als ob man gegen 
eine Mauer läuft, so stark war der Gegensatz zu der durch Klimaanlagen kontrollierten Luft innerhalb 
der Station. Naiverweise dachte ich, wir würden einfach den erstbesten Exit nehmen und uns dann zu Fuß 
bis zu unserem Hostel durchschlagen. Nach knapp 15 Minuten mussten wir dann aber einsehen, dass es 
so keinen Weg bis zur Nathan Road, dem Backpacker-Bereich Hongkongs, gibt. Also zurück in die Station 
und neu orientieren. Mit Hilfe eines Touristens, dem unsere Orientierungslosigkeit wahrscheinlich nur 
allzu bekannt vorkam, haben wir den richtigen Exit genommen und uns mit unseren Rucksäcken aufgemacht. 
Der erste Eindruck Hongkongs ist zum einen der Straßenlärm und zum anderen die interessante Mischung 
aus asiatischem Flair mit starken westlichen Einflüssen. Besonders auffallend sind die alten, äußerlich 
schlecht gepflegten Wohnblöcke in denen im Erdgeschoss gleich mehrere Läden zu finden sind. In so einem 
Gebäude ist auch unser Hostel. Über mehrere Stockwerke verteilt sind augenscheinlich alte Appartments in 
sehr kleine Wohneinheiten eingeteilt und in einem dieser werden wir die nächsten drei Tage verbringen.

5. September 2018, 22:14 im Hostelzimmer, die Klimaanlage läuft und die gewaschenen Klamotten hängen im 
Zimmer

Heute hatten wir den ersten vollen Tag in Hongkong. Der Plan war diesen Tag auf Hongkong Island zu verbringen, 
wo das historische Center von Hongkong ist sowie der Financial/Business District. Wir sind so gegen halb 
10 aus dem Hostel raus gegangen und die warm-feuchte Luft ist uns schon entgegen geschlagen. Es waren schon 
über 30 Grad und sicherlich 75\% Luftfeuchtigkeit. Schon nach wenigen Sekunden hat man überall am Körper 
geklebt. Wir sind über einen kleinen Umweg zur Fähre gegangen und haben uns auf dem Weg etwas zum Frühstücken 
gekauft. Blandine war allerdings von den deftigen Gerüchen in der Frühe nicht wirklich angetan und hat sich 
nur eine Banane gekauft. Ich denke der französische Magen bevorzugt am Morgen etwas leichtes und süßes. 
An der Fähre angekommen war alles relativ unkompliziert und wir sind direk nach Hongkong Island übergesetzt. 
Dort sind wir etwas durch die Straßen geschländert. Die modernen Wolkenkratzer sind wirklich beeindruckend, 
aber von der Stadt an sich war ich nicht wirklich begeistert. Sie lädt irgendwie nicht dazu ein zu Fuß 
entdeckt zu werden. Ein um's andere Mal schreckt man innerlich zusammen, wenn man auf dem schmalen Gehweg 
gehend die Doppeldeckerbusse im Nacken spürt. Allgemein finde ich den Verkehr erdrückender als in anderen 
Städten, häufig gibt es mehrspurige Straßen und Bausstellen. Dem entsprechend finden sich nicht viele Orte, 
an denen man sich in Ruhe nach draußen setzen kann. Sehr schön sind allerdings die vielen alten Bäume und 
Parks. Diese sind sehr schön angelegt und gepflegt. Schade nur, dass Tiere in teilweise erbärmlichen 
Zuständen gehalten werden. 
Am Nachmittag sind wir mit einer Bahn zum Victoria Peak hoch gefahren. Eine der Hauptatraktionen von 
Hongkong. Die Aussicht war beeindruckend, wenn auch etwas diesig, da die Wolken über die Stadt gezogen sind.
Wir haben viel Zeit im Victoria Peak Garden verbracht und uns etwas ausgeruht. Dann ging es zu Fuß zurück 
in die Stadt um etwas zu Essen zu suchen, wenn möglich Dim Sum, eine chinesische Spezialität. Wir haben 
ein schönes Lokal gefunden und es uns schmecken lassen. Die Überraschung war, dass \emph{phoenix tallon} 
sich als Hühnerfuß heraus gestellt hat. Nach anfänglichem Zögern wurde dieser von mir verspeist, auch wenn 
da nichts außer Haut dran ist. 
Auf dem Rückweg zum Hostel haben wir uns dann noch die Lightshow, die es wohl jeden Abend gibt, vom Hafen aus 
angeschaut. Diese war unser Meinung nach überhaupt nichts besonderes, aber wenn man schon mal in der Gegend ist.

\end{document}