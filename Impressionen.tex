\documentclass[11pt]{book}
\usepackage[utf8]{inputenc}
\title{Impressionen}
\author{Alexander Böhm}
\begin{document}
\maketitle
\chapter{Vorbereitungen}
1. September 2019, 12:04 im Zug von München nach Paris, es ist grau und regnet leicht

Wir sitzen im Zug. Nach sehr anstrengenden letzten Tagen haben wir es geschafft mit voll gepacktem
Rucksack aufzubrechen. Kurz vor der Abreise wurde es, wie immer, stressig. Es gibt einfach zu vieles
Dinge die man zu erledigen hat, Wohnung leer räumen, sich von Freunden verabschieden, Arzttermine, 
Behördengänge, und bei all dem nicht den Kopf verlieren während man panisch versucht nichts zu 
vergessen. Ich hatte erwartet, dass sich die Erleichterung, das Gefühl der Entspannung, im Zug 
einstellen würden. Aber ich habe ab dem Zeitpunkt ab dem ich die Wohnung verlassen habe auch den 
Stress hinter mir gelassen. Im Moment empfinde ich nur sehr große Vorfreude, eine schöne Mischung
aus positiver Nervosität und Aufregung. Dies führt dazu, dass ich trotz der körperlichen Müdigkeit
nicht in der Lage zu schlafen. Blandine und ich sind beide froh, dass wir noch 2 Tage in
Paris verbringen können bevor wir wirklich nach Hongkong aufbrechen. Ich denke, dass so ein verzögerter Start doch 
einige Ruhe mit sich bringt, da man nicht direkt zum Flughafen hetzen muss. 

3. September 2018, 14:45 am Gate, die Sonne scheint

Wir sitzen am Gate C91 und der A380-800 wartet schon auf uns. Das Gepäck ist abgegeben, zwei gut 
gepackte Rucksäcke die vom Gewicht etwas ungleich verteilt sind. Meiner wiegt 17 kg während 
Blandines bei ungefähr 11kg eingecheckt wurde. Das wird im Laufe der Reise sicherlich noch etwas 
besser aufgeteilt werden, aber mein Rucksack wirkt jetzt schon fast leer. Wir haben wohl zu 
effizient gepackt. Blandine wirkt sehr entspannt und ruhig, auch wenn sie innerlich sehr aufgeregt 
ist. Ihre Vorfreude auf diese Reise ist ihr die ganze Zeit anzumerken. Ich freue mich so sehr mit 
ihr diese Erfahrung zu machen. 

\chapter{China}

4 September 2018, 21:30 im Traveller Friendship Hostel, Hong Kong

Der eigentliche Beginn der Reise ist für mich heute, als ich das erste Mal ungefilterte Luft in 
Hongkong eingeatmet habe. Es war unerwartet heiß und stickig. Blandine und ich sind nach halbstündiger
Fahrt vom Flughafen an der Kowloon-Station ausgestiegen. Nachdem wir uns mit der Hilfe von drei 
Rolltreppen an die Oberfläche gekämpft haben, waren wir endlich in Hongkong. Es war als ob man gegen 
eine Mauer läuft, so stark war der Gegensatz zu der durch Klimaanlagen kontrollierten Luft innerhalb 
der Station. Naiverweise dachte ich, wir würden einfach den erstbesten Exit nehmen und uns dann zu Fuß 
bis zu unserem Hostel durchschlagen. Nach knapp 15 Minuten mussten wir dann aber einsehen, dass es 
so keinen Weg bis zur Nathan Road, dem Backpacker-Bereich Hongkongs, gibt. Also zurück in die Station 
und neu orientieren. Mit Hilfe eines Touristens, dem unsere Orientierungslosigkeit wahrscheinlich nur 
allzu bekannt vorkam, haben wir den richtigen Exit genommen und uns mit unseren Rucksäcken aufgemacht. 
Der erste Eindruck Hongkongs ist zum einen der Straßenlärm und zum anderen die interessante Mischung 
aus asiatischem Flair mit starken westlichen Einflüssen. Besonders auffallend sind die alten, äußerlich 
schlecht gepflegten Wohnblöcke in denen im Erdgeschoss gleich mehrere Läden zu finden sind. In so einem 
Gebäude ist auch unser Hostel. Über mehrere Stockwerke verteilt sind augenscheinlich alte Appartments in 
sehr kleine Wohneinheiten eingeteilt und in einem dieser werden wir die nächsten drei Tage verbringen.

5. September 2018, 22:14 im Traveller Friendship Hostel, Hong Kong

Heute hatten wir den ersten vollen Tag in Hongkong. Der Plan war diesen Tag auf Hongkong Island zu verbringen, 
wo das historische Center von Hongkong ist sowie der Financial/Business District. Wir sind so gegen halb 
10 aus dem Hostel raus gegangen und die warm-feuchte Luft ist uns schon entgegen geschlagen. Es waren schon 
über 30 Grad und sicherlich 75\% Luftfeuchtigkeit. Schon nach wenigen Sekunden hat man überall am Körper 
geklebt. Wir sind über einen kleinen Umweg zur Fähre gegangen und haben uns auf dem Weg etwas zum Frühstücken 
gekauft. Blandine war allerdings von den deftigen Gerüchen in der Frühe nicht wirklich angetan und hat sich 
nur eine Banane gekauft. Ich denke der französische Magen bevorzugt am Morgen etwas leichtes und süßes. 
An der Fähre angekommen war alles relativ unkompliziert und wir sind direk nach Hongkong Island übergesetzt. 
Dort sind wir etwas durch die Straßen geschländert. Die modernen Wolkenkratzer sind wirklich beeindruckend, 
aber von der Stadt an sich war ich nicht wirklich begeistert. Sie lädt irgendwie nicht dazu ein zu Fuß 
entdeckt zu werden. Ein um's andere Mal schreckt man innerlich zusammen, wenn man auf dem schmalen Gehweg 
gehend die Doppeldeckerbusse im Nacken spürt. Allgemein finde ich den Verkehr erdrückender als in anderen 
Städten, häufig gibt es mehrspurige Straßen und Bausstellen. Dem entsprechend finden sich nicht viele Orte, 
an denen man sich in Ruhe nach draußen setzen kann. Sehr schön sind allerdings die vielen alten Bäume und 
Parks. Diese sind sehr schön angelegt und gepflegt. Schade nur, dass Tiere in teilweise erbärmlichen 
Zuständen gehalten werden. 
Am Nachmittag sind wir mit einer Bahn zum Victoria Peak hoch gefahren. Eine der Hauptatraktionen von 
Hongkong. Die Aussicht war beeindruckend, wenn auch etwas diesig, da die Wolken über die Stadt gezogen sind.
Wir haben viel Zeit im Victoria Peak Garden verbracht und uns etwas ausgeruht. Dann ging es zu Fuß zurück 
in die Stadt um etwas zu Essen zu suchen, wenn möglich Dim Sum, eine chinesische Spezialität. Wir haben 
ein schönes Lokal gefunden und es uns schmecken lassen. Die Überraschung war, dass \emph{phoenix tallon} 
sich als Hühnerfuß heraus gestellt hat. Nach anfänglichem Zögern wurde dieser von mir verspeist, auch wenn 
da nichts außer Haut dran ist. 
Auf dem Rückweg zum Hostel haben wir uns dann noch die Lightshow, die es wohl jeden Abend gibt, vom Hafen aus 
angeschaut. Diese war unser Meinung nach überhaupt nichts besonderes, aber wenn man schon mal in der Gegend ist.

6. September 2018, 22:00, im Apple Inn Hostel, Hong Kong

Heute war ein richtig guter Tag. Wir sind mit der MTR und dem Bus nach Ngong Ping gefahren um uns den Tian Tan 
Buddha anzschauen. Schon die Busfahrt vom Tung Chung nach Ngong Ping war super. Die meisten Touristen nehmen 
eine Seilbahn (die HKD 220 kostet) um dort hin zu kommen. Aber wir als low budget Touristen haben die HKD 17 
Variante bevorzugt. Ich habe es genossen, die Landschaft, die kleinen Dörfer am Wegesrand und auch das 
Gefängnis zu sehen, währen wir deutlich schneller als erwartet am Zielort angekommen sind. Kaum aus dem 
Bus ausgestiegen hat man auch schon die riesengroßen Buddha über sich gesehen. Sitzend und über 35m groß 
beeindruckt er ab dem ersten Augenblick. Der Eintritt ist frei, aber man muss sich die Nähe zum Buddha trotzdem 
verdienen, 220 Treppenstufen mussten wir erklimmen, als wir uns als erste Touristen des Tages in Richtung 
Buddha bewegt haben. Der Anblick war grandios und wir hatten zusätlich eine schöne Aussicht über die Umgebung.
Inseln, die langsam vom Morgennebel frei gegeben werden, von Wolken umringte Berge und die farbenfrohe Monastry 
hielten meine Augen für einige Zeit gefangen. Diese Monastry haben wir uns dann nach dem Buddha noch etwas 
näher angeschaut. Man sieht, dass es nicht alt ist, aber nichtsdestotrotz war es schön anzuschauen. 
Danach sind wir wieder in Richtung Stadt aufgebrochen. Auf dem Weg sind wir fast in der Metro erfroren, weil 
die Klimaanlage dort wohl auf Hochtouren lief. Den Nachmittag haben wir dann im History Museum von Hongkong 
ausklingen lassen, bevor wir abend noch etwas durch den Kowloon Park geschlendert sind.

8. September 2018, Shenzhen 23:50, im bisher besten Airbnb meines Lebens

Heute haben sind wir ins das wirkliche China eingereist. Shenzhen ist eine Stadt direkt neben Hong Kong und so 
konnten wir mit der MTR bis an die Grenze fahren. Die Einreise hat sich etwas hingezogen aber das gehört dazu.
Der Morgen war etwas stressig, weil wir spät dranne waren. Blandine mag das überhaupt nicht und war dem 
entsprechend gestresst. Aber wir haben alles gut gemeistert und waren mit etwas Verspätung an der an der 
Metro-Station ausgestiegen um uns zum Airbnb aufzumachen. Wir wussten nicht genau wo wir hinmussten, weil Google
Maps keine große Hilfe in China ist und Baidu Maps ungefähr 10 verschiendene Orte vorgeschlagen hat (wenn auch alle 
in der Nähe). Glücklicherweise hat mein Freund Zida schon auf uns gewartet und die Lage vor Ort für uns etwas 
ausgekundschaftet. Wir sind dann zu dritt zu dem Appartment gegangen wo wir uns einem Pärchen, Washion und Lyn, 
begrüßt wurden. Die Wohnung ist wirklich sehr schön, besonders nachdem wir die ersten 4 Nächte in Hostels in 
Hongkong verbracht haben. Unsere Hosts waren unfassbar freundlich und haben uns ihre Metrokarten gegeben, damit 
wir uns einfach in der Stadt bewegen können. Wir haben dann den restlichen Tag mit Zida in Shenzhen verbracht. 
Die Stadt ist erst 40 Jahre alt und das sieht man auch. Alles ist sehr neu und modern errichtet. Es gibt
offensichtlich keine Altstadt sondern eher große Promenaden mit vielen Geschäften und groß angelegte Parks. 
Nach dem Lärm und der Enge von Hong Kong hat das aber Blandine und mir sehr gut gefallen. So sind wir mit 
Zida ein bisschen durch die Stadt geschlendert und haben uns viel unterhalten. Zida hat uns mit vielem sehr 
geholfen. Unter anderem haben wir mit seiner Hilfe unsere Zugfahrt für morgen nach Guilin gekauft und auch
eine SIM-Karte, sodass wir Internet in China haben werden. Abends sind wir dann wieder zurück zum Hostel, 
wo unsere Hosts uns wieder begrüßt haben. Wir haben dann noch einen sehr schönen Abend mit ihnen verbracht. 
Sie haben für uns ein Taxi gebucht für 6 Uhr morgens, weil wir nur noch einen Zug um 7:21AM nehmen konnten 
und Washion will sogar mit uns um 5:30 aufstehen um sicher zu stellen, dass das Taxi für uns da sein wird. Es 
ist wirklich unglaublich wie nett und zuvorkommend die beiden sind. Sie haben uns auch gleich Frühstück für den 
Zug mitgegeben. Den Abend haben wir uns dann noch viel unterhalten, viel gelacht und auch viele Selfies mit 
verschiedensten Snapchat-Filtern gemacht. Natürlich war es kein Snapchat sondern die chinesische Variante davon, 
dessen Namen ich schon wieder vergessen habe. Heute war unser erster Tag an dem wir mit Locals wirklich in Kontakt 
getreten sind und es hat sich wieder mal gezeigt wie wertvoll diese Erfahrungen sind. Es ist so erfrischend 
ein Gespräch zu führen und das gegenseitige aufrichtige Interesse zu spüren. Besonders Blandine hat so viel 
Freude ausgestrahlt und ich bin froh, dass wir schon so früh diese Erfahrung sammeln konnten, weil ich ihr davon 
immer vorgeschwärmt habe. 
In meiner Freude habe ich zuerst von dem heutigen Tag geschrieben aber natürlich haben wir auch viel gestern, 
an unserem letzten vollen Tag in Hong Kong erlebt. Am Anfang des Tages mussten wir in ein neues Hostel ziehen, 
welches aber gleich um die Ecke war. Das lief alles reibungslos und wir sind wieder auf die Hong Kong Island 
gefahren um die von meinem Vater empfohlene Ding Ding Tour zu machen. Ding Dings sind doppelstöckige Trams, 
die so wohl einzigartig sind und es war wirklich ein schönes Erlebnis die Stadt auf diese Art zu durchqueren.
Anfangs etwas wackelig und am Ende ziemlich heiß aber trotzdem ein sehr cooles Erlebnis. An der Endhaltestelle 
sind haben Blandine und ich einen kleinen Markt entdeckt, auf dem wir uns eine Pomele, eine Drachenfrucht und 
ein paar Bananen gekauft haben, unser Mittag. Auf dem Weg zu einem ruhigen Plätzchen haben wir dann noch einen 
Laden entdeckt der eine Art Ban Bao angeboten hat, gedämpfte Teigtaschen, die mit Glasnudeln und Pilzen gefüllt 
sind. Das hat mich natürlich besonders gefreut weil ich so etwas schon seit meinem ersten Vietnambesuch 1996 
liebe. Nach dem Mittag sind wir dann mit dem Bus in die Mitte von Hong Kong Island gefahren um einen Teil des 
Wilson Trails zu wandern, bis Repulse Bay Beach. Es gab eine Menge Treppen und der Schweiß lief schon nach wenigen 
Sekunden. Nach circa 90 Minuten gab es die Wahl, nach rechts Richtung Repulse Bay Beach und etwas Entspannung, oder 
geradeaus einen Gipfel nach oben der über 300 Meter über uns war. Wir haben natürlich noch den Gipfel mitgenommen 
um danach zum Strand zu gehen, man muss sich das ja auch verdienen. Der Weg zum Gipfel waren ca 1000 Treppenstufen 
und hat uns einiges abverlangt. Aber wir haben es geschafft und die Belohnung war wirklich erst die Abkühlung im 
Meer, weil es absolut gar keine Aussicht aufgrund des dichten Bewuchses auf dem Gipfel gab. Am Strand sind wir 
erst einmal eine kurze Runde ins Wasser gegangen bevor ich mich meinem Kung Fu gewidmet habe. Ich habe mir gesagt 
immer wenn es gut möglich ist möchte ich mein Kung Fu üben und am dem ersten Strandaufenthalt habe ich das gut 
umsetzen können. Auch wenn ich natürlich schnell viele Zuschauer hatte. Nach 2 Stunden Strand war es danach auch 
schon relativ dunkel und wir haben uns auf dem Weg zurück in die Stadt gemacht um noch etwas zu essen bevor 
wir erschöpft ins Bett gefallen sind. 

9. September 2018, 17:20 auf der Rooftop-Terasse des This Old Place, Guiling

Heute sind wir von Shenzhen nach Guilin gereist. Der Plan war eigentlich einen Zug gegen Mittag zu nehmen 
aber wir hatten gestern fest gestellt, dass es nur noch Tickets für den 7:21AM Zug gab. Also hieß es heute früh 
zeitig aufstehen und ab zum Bahnhof. Das hat alles einwandfrei funktioniert, auch wenn wir nicht in das Taxi 
gestiegen sind, welches für uns von unserem Airbnb Host reseriert wurde. Wir haben also etwas Geld verschwendet 
aber halb so wild. Den Zug haben wir mehr als rechtzeitig erwischt, wir waren über eine Stunde vor Abfahrt am 
Bahnhof. Für die erste Zugfahrt in China dachten wir, sicher ist sicher. Blandine konnte sich aber trotzdem erst 
beruhigen, als wir vor dem Gleis waren, oder besser gesagt in der Wartehalle mit dem Blick auf den Check-In 
Bereich unseres Gleises. Die Zugfahrt war äußerst angenehm. Trotz Tickets der günstigsten Kategorie, haben wir
sehr bequem gesessen während wir mit über 250km/h durch die chinesische Landschaft gedüst sind. Nach ungefähr 
3 Stunden sind wir dann in Guiling angekommen, wo wir ehr zügig den Zug Richtung Innenstadt und zu unserem 
Hostel gefunden haben. Der erste Eindruck von Guilin unterscheidet sich erheblich von Hong Kong und auch 
Shenzhen. Es erinnert mich etwas mehr an Hanoi, aufgrund der Vielzahl an Mopeds. Allerdings ist es viel ruhiger
da die meisten Roller einen Elektroantrieb haben. 
Unser Hostel ist direkt am Li River gelegen. Für 16 Euro die Nacht ist es wirklich ein sehr schönes Hostel, 
mit einer großen Dachterasse, auf der ich gerade sitze. Die Aussicht ist phenomänal. Zum Mittag sind wir vom 
Hostel aus nur ein paar Minuten entlang des Flusses gegangen bis wir eine kleine Küche gefunden haben, die 
natürlich nur auf Chinesisch Dinge angeboten hat. Augenscheinlich war es aber eine Nudelsuppe mit Fleisch, 
genau mein Ding. Wir haben dort also 2 Portionen für je 9 Yuan (ca. 1.15 Euro) bestellt. Mir hat es sehr gut 
gemschmeckt, aber Blandine ist glaube ich immer noch etwas skeptisch was das Essen angeht. Den Nachmittag haben wir 
damit verbracht durch die Stadt zu schlendern. Es gibt mehrere Seen, die miteinander verbunden sind und die 
ganze Anlage ist sehr liebevoll und schön hergerichtet. Überall gibt es schattige Sitzgelegenheiten die schöne 
Aussichten auf kunstvolle Brücken mit den charakteristischen Bergen im Hintergrund erlauben. Während unseres 
Spazierganges haben uns wie üblich in China die Blicke verfolgt. Die Leute sind wohl viele Touristen nicht 
gewohnt. Hier kam es aber auch häufiger vor, dass uns die Menschen lächelnd \emph{hello} zugerufen haben. So 
kam es auch, dass wir vom einem Herren mittleren Alters angesprochen wurden, der Englisch-Lehrer an einer 
hiesigen Schule ist. Mit ihm haben wir uns eine gute halbe Stunde unterhalten während wir den Fluss entlang 
gegangen sind. Er hat uns viele Tips gegeben um den Touristenfallen zu entgehen. Es zeigt sich also wieder 
einmal mehr. Wer den Kontakt zu den Einheimischen nicht scheut, hat viele Vorteile!
Den Rest des Tages werden wir eher ruhig angehen, da wir beide recht müde sind. Die kurze Nacht und das viele
Gehen der letzten Tage machen sich nun wirklich bemerkbar. Daher liegt Blandine auch schon hinter mir auf einer
Holzliege und macht die Augen zu.

11. September 2018, 23:01 auf der Dachterasse des The Old Place, Guilin

Wir haben zwei ereignisreiche Tage hinter uns. Gestern haben wir eine Tagestour nach Yangshuo gemacht. Die 
Stadt liegt inmitten der Kalksteinberge, die den Horizont von Guilin dominieren. Die Hinfahrt war schon sehr 
abenteuerlich für uns beide. Der Plan war, den Bus von dem Hauptbahnhof zu nehmen aber den haben wir nirgends 
gefunden. Auch konnte uns keiner der Einheimischen helfen, da die Sprachbarriere ihre volle Kraft entfaltet hat. 
Nach eine gewissen Wartezeit hat eine ältere Frau uns gesagt sie bringt uns zum Bus der nach Yangshuo fährt, für 
einen Preis von 35 Yuan pro Person, 10 Yuan teurer als wir recherchiert hatten. Das war es uns aber wert, weil 
wir ansonsten nicht gewusst hätten wie wir an unser Ziel gelangen könnten. Nach 10 Minuten Fußweg hat sie uns 
dann in ein Tuc Tuc gesetzt. Wir waren etwas überrascht aber alles ging sehr schnell und schwupps waren wir 
auf einer stark befahrenen Straße. Drei Chinesen saßen mit uns in dem Tuc Tuc und waren genauso überrascht über 
Situation wie wir. Wir dachten schon, das es recht anstrengend werden könnte die Strecke in dieser Art zu 
bewältigen, wo es doch mit einem Bus schon 90 Minuten dauert. Aber nach ungefähr einer halben Stunde hielten wir
an und wurden in einen Bus gesetzt. Es hat also alles super gepasst und wir haben diese Erfahrung sehr genossen.
In Yangshuo angekommen haben wir uns dann Fahrräder ausgeliehen und haben uns auf den Weg nach Xingping gemacht.
Unsere geplante Strecke war aber etwas unangenehm mit dem Fahrrad zu absolvieren, weil es an einer stark befahrenen 
Straße entlang geführt hat. Wir haben also spontan eine andere Route ausgewählt. Diese war viel ruhiger und schöner, 
auch wenn wir uns teilweise durch dichtbewachsene Stellen kämpfen mussten, die definitiv nicht für Fahrräder 
vorgesehen waren. Belohnt für unsere Mühen wurden wir in Xingping mit einer wunderschönen Aussicht. Die dicht 
bewachsenen Spitzen der Kalksteinberge bestimmten den gesamten Blick. Der Himmel war bewölkt und die Aussicht war 
etwas nebelig, sodass die hinteren Bergspitzen immer schemenhafter zu sehen waren. Das hat dem Ganzen einen sehr 
mythischen Ausdruck verliehen. Die Zeit wurde dann schon etwas knapp und wir sind den Rückweg die ursprünglich 
geplante Strecke, der Straße entlang gefahren. Dann haben wir den Bus zurück nach Guilin genommen.
Der heutige Tag war ebenso schön. Um uns den Stress der Bussuche zu ersparen haben wir Tickets im Hostel gekauft. 
Diese waren minimal teurer als nur die Bustickets und dieser Aufpreis hat sich auf jeden Fall gelohnt. So konnten 
wir uns deutlich entspannter nach Longji aufmachen um die Reisterassen zu bewundern. Nach etwas mehr als zwei 
Stunden Busfahrt sind wir dort angekommen. Unser Busfahrer hat gefühlt jedes Fahrzeug auf dieser Strecke überholt 
aber wir waren von seinen Drängel- und Fahrkünsten beeindruckt. Die Reisterassen an sich waren auch wunderschön. 
Blandine und ich sind vier Stunden durch diese atemberaubende menschengemachte Landschaft gewandert. Hunderte 
schmale Reisterassen, die übereinander angelegt wurden und sich die einzelnen Berge entlang wenden sieht man 
nicht überall. Nach jeder Abzweigung hat man eine neue Aussicht von der man am liebsten 20 Fotos machen möchte. 
Zum Mittag haben wir uns eine \emph{bamboo rice} gegönnt. Klebreis der in einen Bambusrohr gefüllt wird und über 
dem offenen Feuer erwärmt wird. Sehr zu empfehlen! Um vier Uhr nachmittags sind wir dann wieder Richtung Guilin 
gefahren wo wir sehr gut zu Abend gegessen haben. Blandine war bisher eher skeptisch, was das chinesische Essen 
angeht, aber heute hat es ihr auch Richtung gut geschmeckt. Wir waren in einem Restaurant ohne englisches Menu. 
Aber am Vorabend war es sehr gut besucht und voller Chinesen. Es musste also gut sein. Und diesen Erwartungen 
hat es entsprochen. Das war unser letzter voller Tag in Guilin. Morgen abend fliegen wir nach Kunming. Den Tag 
nutzen wir zum entspannen und recherchieren.

15. September 2018, 14:40 im Innenhof vom Upland Youth Hostel, Kunming

Die letzten beiden Tage haben wir in Kunming verbracht. Die Anreise war etwas anstrengend, weil unser Flug aus 
Guilin aufgrund schlechten Wetters Verspätung hatte. So hat sich die geplante Ankunftszeit um zwei Stunden auf 
1 Uhr nachts verzögert. Es hat aber trotzdem alles gepasst und wir sind gegen halb 3 an unserem Hostel angekommen.
Dort gab es allerdings einige Probleme mit unserer Reservierung. Wir hatten für zwei Personen in einem Dormitory
gebucht, aber letztendlich war nur ein Bett gebucht. Das ist etwas missverständlich in Agoda angegeben. So 
mussten wir noch für Blandine ein Bett buchen und wir waren in zwei verschiedenen Dormitories. Alles nicht so 
optimal, aber die Betten und Duschen waren sauber, also alles in Ordnung. 
Den nächsten Tag haben wir damit verbracht etwas Kunming zu erkunden. Das Hostel liegt direkt neben dem Green 
Lake. Dort sind mehrere Wege und kleine Plätze über den See angelegt und man kann schön an den von Seerosen 
bedeckten See entlang spazieren. Dort haben wir auch eine kleine Gruppe an Männern gefunden, die offensichtlich 
etwas Kung Fu praktiziert haben. Ich habe natürlich Kontakt aufgenommen und habe so ein bisschen mit ihnen geübt. 
Eines meiner persönlichen Highlights bisher! Die Übungen waren zu zweit und man hat Kraft auf seinen Partner ausgeübt 
und ihn so versucht aus dem Gleichgewicht zu bringen. Das haben die älteren Herren bei mir auch häufig 
geschafft. Aber ich habe mich nicht so schlecht angestellt und wenn ich sauber in meinen Stellungen war, konnten 
sie mich nicht so einfach bewegen. Sie waren sichtlich beeindruckt und erfreut von meinen Fähigkeiten. Ich 
hoffe, dass wir noch häufiger so etwas in China finden werden. Nach dieser tollen Erfahrung, die mich auch zum 
Schwitzen gebracht hat, begann der Magen zu knurren. Also haben wir uns etwas zum Essen gesucht. Kurz hinter dem 
See gab es ein Restaurant das komplett voll war mit Chinesen. Natürlich war die Karte nur auf Chinesisch aber 
eine junge Chinesin, die Englisch konnte, hat uns freundlicherweise geholfen bei der Essensauswahl, und was für 
eine Wahl das war. Das mit Abstand beste Essen der Reise bisher. Auch Blandine fand es sehr gut. Drei alte Männer 
haben in der offenen Küche routiniert mit mehreren halbkugelförmigen Pfannen verschiedenste Soßen und Kräuter 
mit den Nudeln vermischt und über offener Flamme zubereitet. 
Am Nachmittag sind wir zum Golden Temple gefahren, eine Tempelanlage auf einem Hügel in Kunming. Die Anlage war nicht 
sonderlich spektakulär und das wir uns etwas verlaufen haben ist das einzige was in Erinnerung bleibt. Glücklicherweise 
haben wir nur die Hälfte des Eintritts bezahlt, weil wir uns als Studenten ausgegeben haben. 
Gestern haben wir uns vorgenommen einen Tagesausflug zum Shilin Stone Forest zu machen. Aufgrund des morgendlichen 
Regens sind wir etwas später als geplant aufgebrochen, aber das war kein Problem. Nach 90 Minuten Busfahrt sind 
wir am Stone Forest angekommen und die relativ teure Sehenswürdigkeit (175 Yuen) hat sich wirklich gelohnt. 
Zahllose 10 bis 20m hohe karge Kalksteinfelsen die aus dem Boden herausragen und der Fantasie viel Spielraum 
lassen Tiere und andere Dinge zu erkennen. Wir mussten uns durch teilweise 40cm schmale Gänge quetschen und 
mehrere Meter in die Tiefe gehen um durch diesen Steinwald zu gehen. Das war wirklich ein Erlebnis und sowas habe 
ich bisher noch nicht gesehen. Eine wirklich beeindruckende Szenerie! Am Ende hat es leider etwas angefangen zu 
regnen. Das hat die ganzen Wege extrem rutschig gemacht und man musste gut aufpassen nicht zu stürzen. Aber 
trotz des Regens haben wir den ganzen Ausflug sehr genossen. 

17. September 2018, 9:45 im Five Elements Hostel, Dali

Vorgestern sind wir von Kunming in Richtung Dali aufgebrochen. Mittlerweile sind wir wirklich routiniert. Relativ 
entpannt sind wir mit etwas Puffer zum Bahnhof gefahren. Auf dem Weg habe ich mir noch eine Art Ofenkartoffel 
von der Straße gekauft. Im Bahnhof angekommen haben wir noch etwas gewartet. Blandine hat ihre \emph{Impressions}
geschrieben und wurde zum zweiten Mal von Chinesen darauf angesprochen. Sie scheinen von ihrer sauberen Handschrift
beeindruckt zu sein. Meiner Meinung nach ein sehr großes Lob aus einem Land wo Kaligraphie fester Bestandteil 
der Kultur ist. Die Zufgfahrt war normal und wir sind pünktlich um 21:30 in Dali angekommen. Wir wussten nicht genau 
wann der letzte Bus in die Altstadt fährt und hatten schon Sorge wir müssten ein Taxi nehmen. Das wäre auch nicht 
sehr teuer gewesen aber trotzdem 25x so viel wie die Busfahrt. Glücklicherweise gab es noch einen Bus und mit dem 
sind wir dann los gefahren. Über \emph{maps.me} haben wir grob abgeschätzt an welcher Haltestelle wir aussteigen 
müssen und von dort waren es noch ungefähr 30 Minuten Fußweg bis zu unserem Hostel. Anfangs ging dieser durch 
eine schlecht beleuchtete enge Gasse, das war etwas einschüchternd. Am Hostel angekommen wurden wir dann mit einem 
kostenlosen Upgrade begrüßt. Wir wurden in ein großen Zimmer mit Badewanne und Gartenblick gebracht. Ein wirklicher 
Luxus und das für 8 Euro pro Nacht. Nicht schlecht! Blandine und ich waren natürlich extrem glücklich über dieses 
tolle Zimmer, besonders nachdem wir in Kunming in getrennten Dormitories schlafen mussten. So haben wir uns noch 
zwei Bier bestellt und fröhlich ein paar Runden Billard gespielt. 
Für den nächsten Morgen hatten wir über das Hostel eine Tour gebucht. Eigentlich nicht unser Ding, aber ich dachte 
das könnte ziemlich cool sein mit einem anderen Pärchen in einem Van um den Erhai Lake zu fahren. Leider war das 
andere Pärchen nicht in unserem Alter und Chinesisch. Beide, sowie der Fahrer, konnten kein einziges Wort Englisch.
Das hat die Tour etwas anstrengender gemacht als erwartet. Auch ansonsten war es nicht so spektakulär und mit unserem 
ursprünglichen Plan mit dem Fahrrad nach Zhou Cheng zu fahren hätten wir auch das Highlight der Van-Tour gesehen.
Dieses Dort ist bekannt für die kunstvoll blau gefärbten Stoffe und vor Ort konnten wir sehen wie diese seit 
jeher produziert werden. Es ist ein wiklich interessantes Verfahren. Der weiße Stoff wird an manchen Stellen mit 
Schnüren eng zusammen gebunden. In dieser Weise kommt beim Eintauchen des Stoffes nicht überall die blaue Farbe 
hin. So entstehen sehr schöne Motive, wobei es beeindruckend ist, wie viele verschiedene Darstellungen möglich sind.
Drei alte Frauen saßen in einer Ecke und haben die Stoffe zum Färben vorbereitet. Es fällt mir schwer zu verstehen, 
wie sie den Überblick behalten können, sodass am Ende das gewünschte Muster erhält. Aber jahrelange Erfahrung 
lässt es mühelos aussehen.
Heute hatten wir geplant auf einen Berg zu wandern und uns dort einen Tempel anzuschauen und die Aussicht auf 
Dali und den See zu genießen. Leider scheint es heute den ganzen Tag zu regnen und so werden wir wohl in unserem 
tollen Hostelzimmer bleiben. Immerhin sind wir nicht in einem engen, schäbigen Zimmer während es draußen nichts zu 
tun gibt.

\noindent
19. September 2018, 17:00 im Flower Theme Hostel, Lijiang

Der Regen am 17. September hat uns fast den ganzen Tag im Hostelzimmer verbringen lassen. Abends sind wir dann aber 
doch noch in die Altstadt gegangen um eine Kleinigkeit zu essen. Danach wollten wir noch eine Runde Billard 
im Hostel spielen. Wir kamen dann aber ins Gespräch mit der Rezeptionistin und einem der Dauergäste, der jeden 
Tag ein T-Shirt vom Mount Everest trug (er selber war aber noch nicht auf dem Gipfel). Die Kommunikation war 
nicht so einfach, weil er kein Wort Englisch konnte aber mit der Hilfe der Rezeptionistin und etwas Gestik ging alles.
Er hat uns Tee angeboten, der in der Region angepflanzt wird. Dieser wird \emph{Kung Fu Tee} genannt und hat wirklich 
sehr gut geschmeckt. Man gießt kochend heißes Wasser auf den Teeblätter in der Teekanne und drückt mit dem Deckel 
leicht auf den Tee. Die Teeblätter dürfen aber nicht zu lange im Wasser sein, da der Tee ansonsten bitter schmecken 
wird. Wie immer bisher war dieses kurzer Intermezzo mit den Einheimischen sehr cool und hat uns beiden sehr viel 
Freude bereitet. So wurde der eher ereignislose Tag zum Ende hin doch noch ganz cool.
Am nächsten Tag hatten wir geplant einen Zug am Nachmittag in Richtung Lijiang zu nehmen. So hatten wir am Vormittag 
noch etwas Zeit um zu den \emph{Wu Wei Si Kloster} zu gehen. Ich hatte im Vorfeld von diesem Kloster gelesen, dachte 
aber das wäre nicht auf unserer Route. So war ich wirklich begeistert und wollte unbedingt dort hin gehen. Der Grund ist, 
dass man dort als Ausländer Kung Fu mit den dortigen Mönchen trainieren kann. Das Kloster ist von dem Hostel 
ungefähr 90 Minuten zu FUß entfernt und so sind wir in der früh dort hin aufgebrochen. Glücklicherweise hat es nicht 
geregnet. Das Kloster ist schlicht aber schön. Es wirkte viel authentischer als die bisherigen Anlagen, die wir 
besichtigt haben. Gleich hinter dem Eingang haben wir auch schon die Kung Fu Schüler gesehen, die sich gerade 
massiert und gedehnt haben. Wir mussten dann ungefähr eine halbe Stunde warten, bis sie mit einem Training begonnen 
haben. Von dem Training war ich jedoch etwas enttäuscht. Der Aufbau war ähnlich zu meinem Training, Aufwärmübungen 
und dann Techniken. Aber es wurde überhaupt nicht auf eine exakte Ausführung der Bewegungen geachtet. Teilweise 
wirkte es sehr flachsig und etwas unkoordiniert. Einer der Trainer, der ältere, hatte eine kleinere Gruppe mit 
denen er eher die Grundstellungen und einfache Armbewegungen durch gegangen ist. Das sah schon eher nach dem aus 
was ich mir vorgestellt hatte. Alles in allem war es wirklich schön dieses Kloster gesehen zu haben, vor allem 
weil es unerwartetes Glück war. Auch wenn ich nicht vollauf begeistert war könnte ich mir vorstellen dort für 
ein paar Wochen zu trainieren. Die Erfahrung muss trotzdem ziemlich cool sein und das Training ist auf jeden Fall 
nicht anspruchsvoller als in München.
Die Zugfahrt nach Lijiang hatte für uns zwei neue Elemente. Zum Einen haben wir kein Ticket vorher gekauft und 
mussten uns daher noch eines am Abfahrtstag besorgen und zum anderen war es ein normaler Zug mit \emph{hard seats}.
Das Ticket kaufen war kein Problem aber wir mussten einen Zug später nehmen als geplant, weil es für den ersten 
Zug keine Tickets mehr gab. Das war aber auch gar nicht so schlimm, weil es ein Problem bei der Security gab. Zum 
ersten Mal in China schien mein Schweizer Taschenmesser ein Problem zu sein. Die Security-Frau hatte ein Problem 
damit, dass man die Klinge arretiert und sie hat es daher als \emph{controlled knife} bezeichnet und wollte es 
und nicht mit auf dem Zug nehmen lassen. Nach viel Hin- und Herdiskutiererei und der Hilfe von Zida haben wir dann 
eine Lösung gefunden. Das Messer wurde einem Schaffner während der Zugfahrt anvertraut und uns dann bei Ankunft 
in Lijiang wieder gegeben. Noch mal Glück gehabt! Das wäre schon äußerst nervig gewesen, dass Messer zu verlieren.
Die Zugfahrt an sich war nicht sonderlich spektakulär. Wir saßen auch nicht auf Holzsitzen wir wir zuerst gedacht 
haben, sondern auf der untersten Ebene in einem Schlafabteil mit 6 Betten. Nicht mega bequem aber vollkommen ausreichend,
dafür dass wir nur 5 Euro pro Person bezahlt haben.
Den ersten Tag in Lijiang haben wir genutzt um nach \emph{Baisha}, einem kleinen Dorf in der Nähe von Lijiang, 
zu fahren. Das war ein wirklich schöner Ausflug und wir haben auch das erste Souvenir der Reise gekauft, ein mit 
Seide gesticktes Bild. Am Nachmittag hat es dann leider wieder angefangen zu regnen, so dass wir zurück zum Hostel 
sind. Heute Abend werden wir dann noch eine Kleinigkeit essen und uns dann auf unseren 2-Tages Trek in der 
\emph{Tiger Leaping Gorge} vorbereiten. Hoffentlich wird es nicht regnen, damit wir das wirklich genießen können.



\end{document}