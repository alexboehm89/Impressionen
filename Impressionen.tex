\documentclass[11pt]{book}
\usepackage[utf8]{inputenc}
\title{Impressionen}
\author{Alexander Böhm}
\begin{document}
\maketitle
\chapter{Vorbereitungen}
1. September 2019, 12:04 im Zug von München nach Paris, es ist grau und regnet leicht

Wir sitzen im Zug. Nach sehr anstrengenden letzten Tagen haben wir es geschafft mit voll gepacktem
Rucksack aufzubrechen. Kurz vor der Abreise wurde es, wie immer, stressig. Es gibt einfach zu vieles
Dinge die man zu erledigen hat, Wohnung leer räumen, sich von Freunden verabschieden, Arzttermine, 
Behördengänge, und bei all dem nicht den Kopf verlieren während man panisch versucht nichts zu 
vergessen. Ich hatte erwartet, dass sich die Erleichterung, das Gefühl der Entspannung, im Zug 
einstellen würden. Aber ich habe ab dem Zeitpunkt ab dem ich die Wohnung verlassen habe auch den 
Stress hinter mir gelassen. Im Moment empfinde ich nur sehr große Vorfreude, eine schöne Mischung
aus positiver Nervosität und Aufregung. Dies führt dazu, dass ich trotz der körperlichen Müdigkeit
nicht in der Lage zu schlafen. Blandine und ich sind beide froh, dass wir noch 2 Tage in
Paris verbringen können bevor wir wirklich nach Hongkong aufbrechen. Ich denke, dass so ein verzögerter Start doch 
einige Ruhe mit sich bringt, da man nicht direkt zum Flughafen hetzen muss. 

3. September 2018, 14:45 am Gate, die Sonne scheint

Wir sitzen am Gate C91 und der A380-800 wartet schon auf uns. Das Gepäck ist abgegeben, zwei gut 
gepackte Rucksäcke die vom Gewicht etwas ungleich verteilt sind. Meiner wiegt 17 kg während 
Blandines bei ungefähr 11kg eingecheckt wurde. Das wird im Laufe der Reise sicherlich noch etwas 
besser aufgeteilt werden, aber mein Rucksack wirkt jetzt schon fast leer. Wir haben wohl zu 
effizient gepackt. Blandine wirkt sehr entspannt und ruhig, auch wenn sie innerlich sehr aufgeregt 
ist. Ihre Vorfreude auf diese Reise ist ihr die ganze Zeit anzumerken. Ich freue mich so sehr mit 
ihr diese Erfahrung zu machen. 

\chapter{China}

4 September 2018, 21:30 im Traveller Friendship Hostel, Hong Kong

Der eigentliche Beginn der Reise ist für mich heute, als ich das erste Mal ungefilterte Luft in 
Hongkong eingeatmet habe. Es war unerwartet heiß und stickig. Blandine und ich sind nach halbstündiger
Fahrt vom Flughafen an der Kowloon-Station ausgestiegen. Nachdem wir uns mit der Hilfe von drei 
Rolltreppen an die Oberfläche gekämpft haben, waren wir endlich in Hongkong. Es war als ob man gegen 
eine Mauer läuft, so stark war der Gegensatz zu der durch Klimaanlagen kontrollierten Luft innerhalb 
der Station. Naiverweise dachte ich, wir würden einfach den erstbesten Exit nehmen und uns dann zu Fuß 
bis zu unserem Hostel durchschlagen. Nach knapp 15 Minuten mussten wir dann aber einsehen, dass es 
so keinen Weg bis zur Nathan Road, dem Backpacker-Bereich Hongkongs, gibt. Also zurück in die Station 
und neu orientieren. Mit Hilfe eines Touristens, dem unsere Orientierungslosigkeit wahrscheinlich nur 
allzu bekannt vorkam, haben wir den richtigen Exit genommen und uns mit unseren Rucksäcken aufgemacht. 
Der erste Eindruck Hongkongs ist zum einen der Straßenlärm und zum anderen die interessante Mischung 
aus asiatischem Flair mit starken westlichen Einflüssen. Besonders auffallend sind die alten, äußerlich 
schlecht gepflegten Wohnblöcke in denen im Erdgeschoss gleich mehrere Läden zu finden sind. In so einem 
Gebäude ist auch unser Hostel. Über mehrere Stockwerke verteilt sind augenscheinlich alte Appartments in 
sehr kleine Wohneinheiten eingeteilt und in einem dieser werden wir die nächsten drei Tage verbringen.

5. September 2018, 22:14 im Traveller Friendship Hostel, Hong Kong

Heute hatten wir den ersten vollen Tag in Hongkong. Der Plan war diesen Tag auf Hongkong Island zu verbringen, 
wo das historische Center von Hongkong ist sowie der Financial/Business District. Wir sind so gegen halb 
10 aus dem Hostel raus gegangen und die warm-feuchte Luft ist uns schon entgegen geschlagen. Es waren schon 
über 30 Grad und sicherlich 75\% Luftfeuchtigkeit. Schon nach wenigen Sekunden hat man überall am Körper 
geklebt. Wir sind über einen kleinen Umweg zur Fähre gegangen und haben uns auf dem Weg etwas zum Frühstücken 
gekauft. Blandine war allerdings von den deftigen Gerüchen in der Frühe nicht wirklich angetan und hat sich 
nur eine Banane gekauft. Ich denke der französische Magen bevorzugt am Morgen etwas leichtes und süßes. 
An der Fähre angekommen war alles relativ unkompliziert und wir sind direk nach Hongkong Island übergesetzt. 
Dort sind wir etwas durch die Straßen geschländert. Die modernen Wolkenkratzer sind wirklich beeindruckend, 
aber von der Stadt an sich war ich nicht wirklich begeistert. Sie lädt irgendwie nicht dazu ein zu Fuß 
entdeckt zu werden. Ein um's andere Mal schreckt man innerlich zusammen, wenn man auf dem schmalen Gehweg 
gehend die Doppeldeckerbusse im Nacken spürt. Allgemein finde ich den Verkehr erdrückender als in anderen 
Städten, häufig gibt es mehrspurige Straßen und Bausstellen. Dem entsprechend finden sich nicht viele Orte, 
an denen man sich in Ruhe nach draußen setzen kann. Sehr schön sind allerdings die vielen alten Bäume und 
Parks. Diese sind sehr schön angelegt und gepflegt. Schade nur, dass Tiere in teilweise erbärmlichen 
Zuständen gehalten werden. 
Am Nachmittag sind wir mit einer Bahn zum Victoria Peak hoch gefahren. Eine der Hauptatraktionen von 
Hongkong. Die Aussicht war beeindruckend, wenn auch etwas diesig, da die Wolken über die Stadt gezogen sind.
Wir haben viel Zeit im Victoria Peak Garden verbracht und uns etwas ausgeruht. Dann ging es zu Fuß zurück 
in die Stadt um etwas zu Essen zu suchen, wenn möglich Dim Sum, eine chinesische Spezialität. Wir haben 
ein schönes Lokal gefunden und es uns schmecken lassen. Die Überraschung war, dass \emph{phoenix tallon} 
sich als Hühnerfuß heraus gestellt hat. Nach anfänglichem Zögern wurde dieser von mir verspeist, auch wenn 
da nichts außer Haut dran ist. 
Auf dem Rückweg zum Hostel haben wir uns dann noch die Lightshow, die es wohl jeden Abend gibt, vom Hafen aus 
angeschaut. Diese war unser Meinung nach überhaupt nichts besonderes, aber wenn man schon mal in der Gegend ist.

6. September 2018, 22:00, im Apple Inn Hostel, Hong Kong

Heute war ein richtig guter Tag. Wir sind mit der MTR und dem Bus nach Ngong Ping gefahren um uns den Tian Tan 
Buddha anzschauen. Schon die Busfahrt vom Tung Chung nach Ngong Ping war super. Die meisten Touristen nehmen 
eine Seilbahn (die HKD 220 kostet) um dort hin zu kommen. Aber wir als low budget Touristen haben die HKD 17 
Variante bevorzugt. Ich habe es genossen, die Landschaft, die kleinen Dörfer am Wegesrand und auch das 
Gefängnis zu sehen, währen wir deutlich schneller als erwartet am Zielort angekommen sind. Kaum aus dem 
Bus ausgestiegen hat man auch schon die riesengroßen Buddha über sich gesehen. Sitzend und über 35m groß 
beeindruckt er ab dem ersten Augenblick. Der Eintritt ist frei, aber man muss sich die Nähe zum Buddha trotzdem 
verdienen, 220 Treppenstufen mussten wir erklimmen, als wir uns als erste Touristen des Tages in Richtung 
Buddha bewegt haben. Der Anblick war grandios und wir hatten zusätlich eine schöne Aussicht über die Umgebung.
Inseln, die langsam vom Morgennebel frei gegeben werden, von Wolken umringte Berge und die farbenfrohe Monastry 
hielten meine Augen für einige Zeit gefangen. Diese Monastry haben wir uns dann nach dem Buddha noch etwas 
näher angeschaut. Man sieht, dass es nicht alt ist, aber nichtsdestotrotz war es schön anzuschauen. 
Danach sind wir wieder in Richtung Stadt aufgebrochen. Auf dem Weg sind wir fast in der Metro erfroren, weil 
die Klimaanlage dort wohl auf Hochtouren lief. Den Nachmittag haben wir dann im History Museum von Hongkong 
ausklingen lassen, bevor wir abend noch etwas durch den Kowloon Park geschlendert sind.

8. September 2018, Shenzhen 23:50, im bisher besten Airbnb meines Lebens

Heute haben sind wir ins das wirkliche China eingereist. Shenzhen ist eine Stadt direkt neben Hong Kong und so 
konnten wir mit der MTR bis an die Grenze fahren. Die Einreise hat sich etwas hingezogen aber das gehört dazu.
Der Morgen war etwas stressig, weil wir spät dranne waren. Blandine mag das überhaupt nicht und war dem 
entsprechend gestresst. Aber wir haben alles gut gemeistert und waren mit etwas Verspätung an der an der 
Metro-Station ausgestiegen um uns zum Airbnb aufzumachen. Wir wussten nicht genau wo wir hinmussten, weil Google
Maps keine große Hilfe in China ist und Baidu Maps ungefähr 10 verschiendene Orte vorgeschlagen hat (wenn auch alle 
in der Nähe). Glücklicherweise hat mein Freund Zida schon auf uns gewartet und die Lage vor Ort für uns etwas 
ausgekundschaftet. Wir sind dann zu dritt zu dem Appartment gegangen wo wir uns einem Pärchen, Washion und Lyn, 
begrüßt wurden. Die Wohnung ist wirklich sehr schön, besonders nachdem wir die ersten 4 Nächte in Hostels in 
Hongkong verbracht haben. Unsere Hosts waren unfassbar freundlich und haben uns ihre Metrokarten gegeben, damit 
wir uns einfach in der Stadt bewegen können. Wir haben dann den restlichen Tag mit Zida in Shenzhen verbracht. 
Die Stadt ist erst 40 Jahre alt und das sieht man auch. Alles ist sehr neu und modern errichtet. Es gibt
offensichtlich keine Altstadt sondern eher große Promenaden mit vielen Geschäften und groß angelegte Parks. 
Nach dem Lärm und der Enge von Hong Kong hat das aber Blandine und mir sehr gut gefallen. So sind wir mit 
Zida ein bisschen durch die Stadt geschlendert und haben uns viel unterhalten. Zida hat uns mit vielem sehr 
geholfen. Unter anderem haben wir mit seiner Hilfe unsere Zugfahrt für morgen nach Guilin gekauft und auch
eine SIM-Karte, sodass wir Internet in China haben werden. Abends sind wir dann wieder zurück zum Hostel, 
wo unsere Hosts uns wieder begrüßt haben. Wir haben dann noch einen sehr schönen Abend mit ihnen verbracht. 
Sie haben für uns ein Taxi gebucht für 6 Uhr morgens, weil wir nur noch einen Zug um 7:21AM nehmen konnten 
und Washion will sogar mit uns um 5:30 aufstehen um sicher zu stellen, dass das Taxi für uns da sein wird. Es 
ist wirklich unglaublich wie nett und zuvorkommend die beiden sind. Sie haben uns auch gleich Frühstück für den 
Zug mitgegeben. Den Abend haben wir uns dann noch viel unterhalten, viel gelacht und auch viele Selfies mit 
verschiedensten Snapchat-Filtern gemacht. Natürlich war es kein Snapchat sondern die chinesische Variante davon, 
dessen Namen ich schon wieder vergessen habe. Heute war unser erster Tag an dem wir mit Locals wirklich in Kontakt 
getreten sind und es hat sich wieder mal gezeigt wie wertvoll diese Erfahrungen sind. Es ist so erfrischend 
ein Gespräch zu führen und das gegenseitige aufrichtige Interesse zu spüren. Besonders Blandine hat so viel 
Freude ausgestrahlt und ich bin froh, dass wir schon so früh diese Erfahrung sammeln konnten, weil ich ihr davon 
immer vorgeschwärmt habe. 
In meiner Freude habe ich zuerst von dem heutigen Tag geschrieben aber natürlich haben wir auch viel gestern, 
an unserem letzten vollen Tag in Hong Kong erlebt. Am Anfang des Tages mussten wir in ein neues Hostel ziehen, 
welches aber gleich um die Ecke war. Das lief alles reibungslos und wir sind wieder auf die Hong Kong Island 
gefahren um die von meinem Vater empfohlene Ding Ding Tour zu machen. Ding Dings sind doppelstöckige Trams, 
die so wohl einzigartig sind und es war wirklich ein schönes Erlebnis die Stadt auf diese Art zu durchqueren.
Anfangs etwas wackelig und am Ende ziemlich heiß aber trotzdem ein sehr cooles Erlebnis. An der Endhaltestelle 
sind haben Blandine und ich einen kleinen Markt entdeckt, auf dem wir uns eine Pomele, eine Drachenfrucht und 
ein paar Bananen gekauft haben, unser Mittag. Auf dem Weg zu einem ruhigen Plätzchen haben wir dann noch einen 
Laden entdeckt der eine Art Ban Bao angeboten hat, gedämpfte Teigtaschen, die mit Glasnudeln und Pilzen gefüllt 
sind. Das hat mich natürlich besonders gefreut weil ich so etwas schon seit meinem ersten Vietnambesuch 1996 
liebe. Nach dem Mittag sind wir dann mit dem Bus in die Mitte von Hong Kong Island gefahren um einen Teil des 
Wilson Trails zu wandern, bis Repulse Bay Beach. Es gab eine Menge Treppen und der Schweiß lief schon nach wenigen 
Sekunden. Nach circa 90 Minuten gab es die Wahl, nach rechts Richtung Repulse Bay Beach und etwas Entspannung, oder 
geradeaus einen Gipfel nach oben der über 300 Meter über uns war. Wir haben natürlich noch den Gipfel mitgenommen 
um danach zum Strand zu gehen, man muss sich das ja auch verdienen. Der Weg zum Gipfel waren ca 1000 Treppenstufen 
und hat uns einiges abverlangt. Aber wir haben es geschafft und die Belohnung war wirklich erst die Abkühlung im 
Meer, weil es absolut gar keine Aussicht aufgrund des dichten Bewuchses auf dem Gipfel gab. Am Strand sind wir 
erst einmal eine kurze Runde ins Wasser gegangen bevor ich mich meinem Kung Fu gewidmet habe. Ich habe mir gesagt 
immer wenn es gut möglich ist möchte ich mein Kung Fu üben und am dem ersten Strandaufenthalt habe ich das gut 
umsetzen können. Auch wenn ich natürlich schnell viele Zuschauer hatte. Nach 2 Stunden Strand war es danach auch 
schon relativ dunkel und wir haben uns auf dem Weg zurück in die Stadt gemacht um noch etwas zu essen bevor 
wir erschöpft ins Bett gefallen sind. 

9. September 2018, 17:20 auf der Rooftop-Terasse des This Old Place, Guiling

Heute sind wir von Shenzhen nach Guilin gereist. Der Plan war eigentlich einen Zug gegen Mittag zu nehmen 
aber wir hatten gestern fest gestellt, dass es nur noch Tickets für den 7:21AM Zug gab. Also hieß es heute früh 
zeitig aufstehen und ab zum Bahnhof. Das hat alles einwandfrei funktioniert, auch wenn wir nicht in das Taxi 
gestiegen sind, welches für uns von unserem Airbnb Host reseriert wurde. Wir haben also etwas Geld verschwendet 
aber halb so wild. Den Zug haben wir mehr als rechtzeitig erwischt, wir waren über eine Stunde vor Abfahrt am 
Bahnhof. Für die erste Zugfahrt in China dachten wir, sicher ist sicher. Blandine konnte sich aber trotzdem erst 
beruhigen, als wir vor dem Gleis waren, oder besser gesagt in der Wartehalle mit dem Blick auf den Check-In 
Bereich unseres Gleises. Die Zugfahrt war äußerst angenehm. Trotz Tickets der günstigsten Kategorie, haben wir
sehr bequem gesessen während wir mit über 250km/h durch die chinesische Landschaft gedüst sind. Nach ungefähr 
3 Stunden sind wir dann in Guiling angekommen, wo wir ehr zügig den Zug Richtung Innenstadt und zu unserem 
Hostel gefunden haben. Der erste Eindruck von Guilin unterscheidet sich erheblich von Hong Kong und auch 
Shenzhen. Es erinnert mich etwas mehr an Hanoi, aufgrund der Vielzahl an Mopeds. Allerdings ist es viel ruhiger
da die meisten Roller einen Elektroantrieb haben. 
Unser Hostel ist direkt am Li River gelegen. Für 16 Euro die Nacht ist es wirklich ein sehr schönes Hostel, 
mit einer großen Dachterasse, auf der ich gerade sitze. Die Aussicht ist phenomänal. Zum Mittag sind wir vom 
Hostel aus nur ein paar Minuten entlang des Flusses gegangen bis wir eine kleine Küche gefunden haben, die 
natürlich nur auf Chinesisch Dinge angeboten hat. Augenscheinlich war es aber eine Nudelsuppe mit Fleisch, 
genau mein Ding. Wir haben dort also 2 Portionen für je 9 Yuan (ca. 1.15 Euro) bestellt. Mir hat es sehr gut 
gemschmeckt, aber Blandine ist glaube ich immer noch etwas skeptisch was das Essen angeht. Den Nachmittag haben wir 
damit verbracht durch die Stadt zu schlendern. Es gibt mehrere Seen, die miteinander verbunden sind und die 
ganze Anlage ist sehr liebevoll und schön hergerichtet. Überall gibt es schattige Sitzgelegenheiten die schöne 
Aussichten auf kunstvolle Brücken mit den charakteristischen Bergen im Hintergrund erlauben. Während unseres 
Spazierganges haben uns wie üblich in China die Blicke verfolgt. Die Leute sind wohl viele Touristen nicht 
gewohnt. Hier kam es aber auch häufiger vor, dass uns die Menschen lächelnd \emph{hello} zugerufen haben. So 
kam es auch, dass wir vom einem Herren mittleren Alters angesprochen wurden, der Englisch-Lehrer an einer 
hiesigen Schule ist. Mit ihm haben wir uns eine gute halbe Stunde unterhalten während wir den Fluss entlang 
gegangen sind. Er hat uns viele Tips gegeben um den Touristenfallen zu entgehen. Es zeigt sich also wieder 
einmal mehr. Wer den Kontakt zu den Einheimischen nicht scheut, hat viele Vorteile!
Den Rest des Tages werden wir eher ruhig angehen, da wir beide recht müde sind. Die kurze Nacht und das viele
Gehen der letzten Tage machen sich nun wirklich bemerkbar. Daher liegt Blandine auch schon hinter mir auf einer
Holzliege und macht die Augen zu.

11. September 2018, 23:01 auf der Dachterasse des The Old Place, Guilin

Wir haben zwei ereignisreiche Tage hinter uns. Gestern haben wir eine Tagestour nach Yangshuo gemacht. Die 
Stadt liegt inmitten der Kalksteinberge, die den Horizont von Guilin dominieren. Die Hinfahrt war schon sehr 
abenteuerlich für uns beide. Der Plan war, den Bus von dem Hauptbahnhof zu nehmen aber den haben wir nirgends 
gefunden. Auch konnte uns keiner der Einheimischen helfen, da die Sprachbarriere ihre volle Kraft entfaltet hat. 
Nach eine gewissen Wartezeit hat eine ältere Frau uns gesagt sie bringt uns zum Bus der nach Yangshuo fährt, für 
einen Preis von 35 Yuan pro Person, 10 Yuan teurer als wir recherchiert hatten. Das war es uns aber wert, weil 
wir ansonsten nicht gewusst hätten wie wir an unser Ziel gelangen könnten. Nach 10 Minuten Fußweg hat sie uns 
dann in ein Tuc Tuc gesetzt. Wir waren etwas überrascht aber alles ging sehr schnell und schwupps waren wir 
auf einer stark befahrenen Straße. Drei Chinesen saßen mit uns in dem Tuc Tuc und waren genauso überrascht über 
Situation wie wir. Wir dachten schon, das es recht anstrengend werden könnte die Strecke in dieser Art zu 
bewältigen, wo es doch mit einem Bus schon 90 Minuten dauert. Aber nach ungefähr einer halben Stunde hielten wir
an und wurden in einen Bus gesetzt. Es hat also alles super gepasst und wir haben diese Erfahrung sehr genossen.
In Yangshuo angekommen haben wir uns dann Fahrräder ausgeliehen und haben uns auf den Weg nach Xingping gemacht.
Unsere geplante Strecke war aber etwas unangenehm mit dem Fahrrad zu absolvieren, weil es an einer stark befahrenen 
Straße entlang geführt hat. Wir haben also spontan eine andere Route ausgewählt. Diese war viel ruhiger und schöner, 
auch wenn wir uns teilweise durch dichtbewachsene Stellen kämpfen mussten, die definitiv nicht für Fahrräder 
vorgesehen waren. Belohnt für unsere Mühen wurden wir in Xingping mit einer wunderschönen Aussicht. Die dicht 
bewachsenen Spitzen der Kalksteinberge bestimmten den gesamten Blick. Der Himmel war bewölkt und die Aussicht war 
etwas nebelig, sodass die hinteren Bergspitzen immer schemenhafter zu sehen waren. Das hat dem Ganzen einen sehr 
mythischen Ausdruck verliehen. Die Zeit wurde dann schon etwas knapp und wir sind den Rückweg die ursprünglich 
geplante Strecke, der Straße entlang gefahren. Dann haben wir den Bus zurück nach Guilin genommen.
Der heutige Tag war ebenso schön. Um uns den Stress der Bussuche zu ersparen haben wir Tickets im Hostel gekauft. 
Diese waren minimal teurer als nur die Bustickets und dieser Aufpreis hat sich auf jeden Fall gelohnt. So konnten 
wir uns deutlich entspannter nach Longji aufmachen um die Reisterassen zu bewundern. Nach etwas mehr als zwei 
Stunden Busfahrt sind wir dort angekommen. Unser Busfahrer hat gefühlt jedes Fahrzeug auf dieser Strecke überholt 
aber wir waren von seinen Drängel- und Fahrkünsten beeindruckt. Die Reisterassen an sich waren auch wunderschön. 
Blandine und ich sind vier Stunden durch diese atemberaubende menschengemachte Landschaft gewandert. Hunderte 
schmale Reisterassen, die übereinander angelegt wurden und sich die einzelnen Berge entlang wenden sieht man 
nicht überall. Nach jeder Abzweigung hat man eine neue Aussicht von der man am liebsten 20 Fotos machen möchte. 
Zum Mittag haben wir uns eine \emph{bamboo rice} gegönnt. Klebreis der in einen Bambusrohr gefüllt wird und über 
dem offenen Feuer erwärmt wird. Sehr zu empfehlen! Um vier Uhr nachmittags sind wir dann wieder Richtung Guilin 
gefahren wo wir sehr gut zu Abend gegessen haben. Blandine war bisher eher skeptisch, was das chinesische Essen 
angeht, aber heute hat es ihr auch Richtung gut geschmeckt. Wir waren in einem Restaurant ohne englisches Menu. 
Aber am Vorabend war es sehr gut besucht und voller Chinesen. Es musste also gut sein. Und diesen Erwartungen 
hat es entsprochen. Das war unser letzter voller Tag in Guilin. Morgen abend fliegen wir nach Kunming. Den Tag 
nutzen wir zum entspannen und recherchieren.

15. September 2018, 14:40 im Innenhof vom Upland Youth Hostel, Kunming

Die letzten beiden Tage haben wir in Kunming verbracht. Die Anreise war etwas anstrengend, weil unser Flug aus 
Guilin aufgrund schlechten Wetters Verspätung hatte. So hat sich die geplante Ankunftszeit um zwei Stunden auf 
1 Uhr nachts verzögert. Es hat aber trotzdem alles gepasst und wir sind gegen halb 3 an unserem Hostel angekommen.
Dort gab es allerdings einige Probleme mit unserer Reservierung. Wir hatten für zwei Personen in einem Dormitory
gebucht, aber letztendlich war nur ein Bett gebucht. Das ist etwas missverständlich in Agoda angegeben. So 
mussten wir noch für Blandine ein Bett buchen und wir waren in zwei verschiedenen Dormitories. Alles nicht so 
optimal, aber die Betten und Duschen waren sauber, also alles in Ordnung. 
Den nächsten Tag haben wir damit verbracht etwas Kunming zu erkunden. Das Hostel liegt direkt neben dem Green 
Lake. Dort sind mehrere Wege und kleine Plätze über den See angelegt und man kann schön an den von Seerosen 
bedeckten See entlang spazieren. Dort haben wir auch eine kleine Gruppe an Männern gefunden, die offensichtlich 
etwas Kung Fu praktiziert haben. Ich habe natürlich Kontakt aufgenommen und habe so ein bisschen mit ihnen geübt. 
Eines meiner persönlichen Highlights bisher! Die Übungen waren zu zweit und man hat Kraft auf seinen Partner ausgeübt 
und ihn so versucht aus dem Gleichgewicht zu bringen. Das haben die älteren Herren bei mir auch häufig 
geschafft. Aber ich habe mich nicht so schlecht angestellt und wenn ich sauber in meinen Stellungen war, konnten 
sie mich nicht so einfach bewegen. Sie waren sichtlich beeindruckt und erfreut von meinen Fähigkeiten. Ich 
hoffe, dass wir noch häufiger so etwas in China finden werden. Nach dieser tollen Erfahrung, die mich auch zum 
Schwitzen gebracht hat, begann der Magen zu knurren. Also haben wir uns etwas zum Essen gesucht. Kurz hinter dem 
See gab es ein Restaurant das komplett voll war mit Chinesen. Natürlich war die Karte nur auf Chinesisch aber 
eine junge Chinesin, die Englisch konnte, hat uns freundlicherweise geholfen bei der Essensauswahl, und was für 
eine Wahl das war. Das mit Abstand beste Essen der Reise bisher. Auch Blandine fand es sehr gut. Drei alte Männer 
haben in der offenen Küche routiniert mit mehreren halbkugelförmigen Pfannen verschiedenste Soßen und Kräuter 
mit den Nudeln vermischt und über offener Flamme zubereitet. 
Am Nachmittag sind wir zum Golden Temple gefahren, eine Tempelanlage auf einem Hügel in Kunming. Die Anlage war nicht 
sonderlich spektakulär und das wir uns etwas verlaufen haben ist das einzige was in Erinnerung bleibt. Glücklicherweise 
haben wir nur die Hälfte des Eintritts bezahlt, weil wir uns als Studenten ausgegeben haben. 
Gestern haben wir uns vorgenommen einen Tagesausflug zum Shilin Stone Forest zu machen. Aufgrund des morgendlichen 
Regens sind wir etwas später als geplant aufgebrochen, aber das war kein Problem. Nach 90 Minuten Busfahrt sind 
wir am Stone Forest angekommen und die relativ teure Sehenswürdigkeit (175 Yuen) hat sich wirklich gelohnt. 
Zahllose 10 bis 20m hohe karge Kalksteinfelsen die aus dem Boden herausragen und der Fantasie viel Spielraum 
lassen Tiere und andere Dinge zu erkennen. Wir mussten uns durch teilweise 40cm schmale Gänge quetschen und 
mehrere Meter in die Tiefe gehen um durch diesen Steinwald zu gehen. Das war wirklich ein Erlebnis und sowas habe 
ich bisher noch nicht gesehen. Eine wirklich beeindruckende Szenerie! Am Ende hat es leider etwas angefangen zu 
regnen. Das hat die ganzen Wege extrem rutschig gemacht und man musste gut aufpassen nicht zu stürzen. Aber 
trotz des Regens haben wir den ganzen Ausflug sehr genossen. 
Am Abend haben wir dann wieder gut gegessen und sind über den Park im See zurück zu unserem Hostel gegangen. 
Wie auch schon in anderen Städten hat sich in dem Park das Abendleben abgespielt. Diesmal gab es aber etwas 
außergewöhnliches zu sehen, eine Art Freiluftdisko. Prinzipiell ein Typ mit einer lauten Soundanlage und MP3-Player.
Gespielt wurden Technolieder aus den frühen 2000ern und getanzt haben hauptsächlich ältere Leute. Am auffälligsten 
war hierbei eine sehr alte Oma, die den Techno offensichtlich ziemlich gut fand und die ganze Zeit getanzt hat.
Das war schon eine lustige Szenerie.

17. September 2018, 9:45 im Five Elements Hostel, Dali

Vorgestern sind wir von Kunming in Richtung Dali aufgebrochen. Mittlerweile sind wir wirklich routiniert. Relativ 
entpannt sind wir mit etwas Puffer zum Bahnhof gefahren. Auf dem Weg habe ich mir noch eine Art Ofenkartoffel 
von der Straße gekauft. Im Bahnhof angekommen haben wir noch etwas gewartet. Blandine hat ihre \emph{Impressions}
geschrieben und wurde zum zweiten Mal von Chinesen darauf angesprochen. Sie scheinen von ihrer sauberen Handschrift
beeindruckt zu sein. Meiner Meinung nach ein sehr großes Lob aus einem Land wo Kaligraphie fester Bestandteil 
der Kultur ist. Die Zufgfahrt war normal und wir sind pünktlich um 21:30 in Dali angekommen. Wir wussten nicht genau 
wann der letzte Bus in die Altstadt fährt und hatten schon Sorge wir müssten ein Taxi nehmen. Das wäre auch nicht 
sehr teuer gewesen aber trotzdem 25x so viel wie die Busfahrt. Glücklicherweise gab es noch einen Bus und mit dem 
sind wir dann los gefahren. Über \emph{maps.me} haben wir grob abgeschätzt an welcher Haltestelle wir aussteigen 
müssen und von dort waren es noch ungefähr 30 Minuten Fußweg bis zu unserem Hostel. Anfangs ging dieser durch 
eine schlecht beleuchtete enge Gasse, das war etwas einschüchternd. Am Hostel angekommen wurden wir dann mit einem 
kostenlosen Upgrade begrüßt. Wir wurden in ein großen Zimmer mit Badewanne und Gartenblick gebracht. Ein wirklicher 
Luxus und das für 8 Euro pro Nacht. Nicht schlecht! Blandine und ich waren natürlich extrem glücklich über dieses 
tolle Zimmer, besonders nachdem wir in Kunming in getrennten Dormitories schlafen mussten. So haben wir uns noch 
zwei Bier bestellt und fröhlich ein paar Runden Billard gespielt. 
Für den nächsten Morgen hatten wir über das Hostel eine Tour gebucht. Eigentlich nicht unser Ding, aber ich dachte 
das könnte ziemlich cool sein mit einem anderen Pärchen in einem Van um den Erhai Lake zu fahren. Leider war das 
andere Pärchen nicht in unserem Alter und Chinesisch. Beide, sowie der Fahrer, konnten kein einziges Wort Englisch.
Das hat die Tour etwas anstrengender gemacht als erwartet. Auch ansonsten war es nicht so spektakulär und mit unserem 
ursprünglichen Plan mit dem Fahrrad nach Zhou Cheng zu fahren hätten wir auch das Highlight der Van-Tour gesehen.
Dieses Dort ist bekannt für die kunstvoll blau gefärbten Stoffe und vor Ort konnten wir sehen wie diese seit 
jeher produziert werden. Es ist ein wiklich interessantes Verfahren. Der weiße Stoff wird an manchen Stellen mit 
Schnüren eng zusammen gebunden. In dieser Weise kommt beim Eintauchen des Stoffes nicht überall die blaue Farbe 
hin. So entstehen sehr schöne Motive, wobei es beeindruckend ist, wie viele verschiedene Darstellungen möglich sind.
Drei alte Frauen saßen in einer Ecke und haben die Stoffe zum Färben vorbereitet. Es fällt mir schwer zu verstehen, 
wie sie den Überblick behalten können, sodass am Ende das gewünschte Muster erhält. Aber jahrelange Erfahrung 
lässt es mühelos aussehen.
Heute hatten wir geplant auf einen Berg zu wandern und uns dort einen Tempel anzuschauen und die Aussicht auf 
Dali und den See zu genießen. Leider scheint es heute den ganzen Tag zu regnen und so werden wir wohl in unserem 
tollen Hostelzimmer bleiben. Immerhin sind wir nicht in einem engen, schäbigen Zimmer während es draußen nichts zu 
tun gibt.

\noindent
19. September 2018, 17:00 im Flower Theme Hostel, Lijiang

Der Regen am 17. September hat uns fast den ganzen Tag im Hostelzimmer verbringen lassen. Abends sind wir dann aber 
doch noch in die Altstadt gegangen um eine Kleinigkeit zu essen. Danach wollten wir noch eine Runde Billard 
im Hostel spielen. Wir kamen dann aber ins Gespräch mit der Rezeptionistin und einem der Dauergäste, der jeden 
Tag ein T-Shirt vom Mount Everest trug (er selber war aber noch nicht auf dem Gipfel). Die Kommunikation war 
nicht so einfach, weil er kein Wort Englisch konnte aber mit der Hilfe der Rezeptionistin und etwas Gestik ging alles.
Er hat uns Tee angeboten, der in der Region angepflanzt wird. Dieser wird \emph{Kung Fu Tee} genannt und hat wirklich 
sehr gut geschmeckt. Man gießt kochend heißes Wasser auf den Teeblätter in der Teekanne und drückt mit dem Deckel 
leicht auf den Tee. Die Teeblätter dürfen aber nicht zu lange im Wasser sein, da der Tee ansonsten bitter schmecken 
wird. Wie immer bisher war dieses kurzer Intermezzo mit den Einheimischen sehr cool und hat uns beiden sehr viel 
Freude bereitet. So wurde der eher ereignislose Tag zum Ende hin doch noch ganz cool.
Am nächsten Tag hatten wir geplant einen Zug am Nachmittag in Richtung Lijiang zu nehmen. So hatten wir am Vormittag 
noch etwas Zeit um zu den \emph{Wu Wei Si Kloster} zu gehen. Ich hatte im Vorfeld von diesem Kloster gelesen, dachte 
aber das wäre nicht auf unserer Route. So war ich wirklich begeistert und wollte unbedingt dort hin gehen. Der Grund ist, 
dass man dort als Ausländer Kung Fu mit den dortigen Mönchen trainieren kann. Das Kloster ist von dem Hostel 
ungefähr 90 Minuten zu FUß entfernt und so sind wir in der früh dort hin aufgebrochen. Glücklicherweise hat es nicht 
geregnet. Das Kloster ist schlicht aber schön. Es wirkte viel authentischer als die bisherigen Anlagen, die wir 
besichtigt haben. Gleich hinter dem Eingang haben wir auch schon die Kung Fu Schüler gesehen, die sich gerade 
massiert und gedehnt haben. Wir mussten dann ungefähr eine halbe Stunde warten, bis sie mit einem Training begonnen 
haben. Von dem Training war ich jedoch etwas enttäuscht. Der Aufbau war ähnlich zu meinem Training, Aufwärmübungen 
und dann Techniken. Aber es wurde überhaupt nicht auf eine exakte Ausführung der Bewegungen geachtet. Teilweise 
wirkte es sehr flachsig und etwas unkoordiniert. Einer der Trainer, der ältere, hatte eine kleinere Gruppe mit 
denen er eher die Grundstellungen und einfache Armbewegungen durch gegangen ist. Das sah schon eher nach dem aus 
was ich mir vorgestellt hatte. Alles in allem war es wirklich schön dieses Kloster gesehen zu haben, vor allem 
weil es unerwartetes Glück war. Auch wenn ich nicht vollauf begeistert war könnte ich mir vorstellen dort für 
ein paar Wochen zu trainieren. Die Erfahrung muss trotzdem ziemlich cool sein und das Training ist auf jeden Fall 
nicht anspruchsvoller als in München.
Die Zugfahrt nach Lijiang hatte für uns zwei neue Elemente. Zum Einen haben wir kein Ticket vorher gekauft und 
mussten uns daher noch eines am Abfahrtstag besorgen und zum anderen war es ein normaler Zug mit \emph{hard seats}.
Das Ticket kaufen war kein Problem aber wir mussten einen Zug später nehmen als geplant, weil es für den ersten 
Zug keine Tickets mehr gab. Das war aber auch gar nicht so schlimm, weil es ein Problem bei der Security gab. Zum 
ersten Mal in China schien mein Schweizer Taschenmesser ein Problem zu sein. Die Security-Frau hatte ein Problem 
damit, dass man die Klinge arretiert und sie hat es daher als \emph{controlled knife} bezeichnet und wollte es 
und nicht mit auf dem Zug nehmen lassen. Nach viel Hin- und Herdiskutiererei und der Hilfe von Zida haben wir dann 
eine Lösung gefunden. Das Messer wurde einem Schaffner während der Zugfahrt anvertraut und uns dann bei Ankunft 
in Lijiang wieder gegeben. Noch mal Glück gehabt! Das wäre schon äußerst nervig gewesen, dass Messer zu verlieren.
Die Zugfahrt an sich war nicht sonderlich spektakulär. Wir saßen auch nicht auf Holzsitzen wir wir zuerst gedacht 
haben, sondern auf der untersten Ebene in einem Schlafabteil mit 6 Betten. Nicht mega bequem aber vollkommen ausreichend,
dafür dass wir nur 5 Euro pro Person bezahlt haben.
Den ersten Tag in Lijiang haben wir genutzt um nach \emph{Baisha}, einem kleinen Dorf in der Nähe von Lijiang, 
zu fahren. Das war ein wirklich schöner Ausflug und wir haben auch das erste Souvenir der Reise gekauft, ein mit 
Seide gesticktes Bild. Am Nachmittag hat es dann leider wieder angefangen zu regnen, so dass wir zurück zum Hostel 
sind. Heute Abend werden wir dann noch eine Kleinigkeit essen und uns dann auf unseren 2-Tages Trek in der 
\emph{Tiger Leaping Gorge} vorbereiten. Hoffentlich wird es nicht regnen, damit wir das wirklich genießen können.

21. September 2018, Flower Theme Hotel, Lijiang

Was für ein Trek. Blandine und ich haben zwei wunderschöne Tage hinter uns, die für uns beide das bisherige Highlight 
der Weltreise sind. Es hat wirklich alles super gepasst. Das Wetter perfekt mitgespielt, kein Regen und wenig 
Wolken. Wir sind mit dem Bus zum Anfang des Treks gefahren und haben schon während der Fahrt einen Chinesen 
kennen gelernt. Er hat uns gefragt ob wir auch den Trek machen und ob wir diesen gemeinsam gehen wollen. Das hat 
uns natürlich gefreut. Wir haben also gemeinsam mit Hung den Trek begonnen. Anfangs war es noch ein entspannter 
Spaziergang, auch wenn es schon ordentlich nach oben ging. Am Wegesrand gab es einige Stände in denen die Naxi,
die Minorität die dort in der Region heimisch ist, die üblichen Dinge angeboten haben. Unüblich war das Weed, 
welchen sie alle im Sortiment hatten. Qualität war ok, aber ich habe nichts gekauft, weil wir auch bald nach 
Chengdu fliegen werden. Günstig wäre es aber gewesen, ein großer Zipper für 50 Yuan. Wir sind also weiter gegangen
und vor uns lagen nun die \emph{28 bends}, 28 Schleifen um sie die Schlucht hinauf zu kämpfen. Im Lonely Planet
wird dieser Trek als sehr schwierig beschrieben, besonders in Bezug auf diese Stelle. Aber es wird dort doch stark 
übertrieben. Oben angekommen gab es dann die ersten von vielen atemberaubende Ausblicke. Die Schlucht lag vor uns,
auf der rechten Seite erhebt sich sehr steil das Massiv des \emph{Jade Dragon Snow Mountain}. Dieses Bergmassiv 
hat 13 Spitzen von denen alle mehr als 4000m in die Höhe ragen. Diese grandiose Anblick wird uns den ganzen Trek 
über begleiten. Die unteren Abschnitte sind dicht bewachsen und bilden eine geschlossene grüne Decke. Zu den Gipfeln
wird es immer karger bis teilweise auch Schnee zu sehen ist. Währenddessen strömt der Yangtse, der aufgrund vorheriger 
Regenfällen braun ist, kräftig durch die Schlucht. 
Unser Ziel für den Tag war das Halfway Guest House. Dort sind wir gegen 16:30 angekommen. Nach dem Tag hieß es 
erst einmal mit einem kalten Bier auf der Dachterasse des Guest Houses entspannen und den Blick auf die massive
Bergwand direkt vor uns genießen. Dort kamen wir mit Tom, einem Israeli ins Gespräch. Wir haben uns super verstanden 
und den Abend gemeinsam mit ihm und Hung verbracht. Tom hatte sich bei einem der Stände ein bisschen Gras gekauft
und so kam ich in den Genuss des lokalen Anbaus auch ohne mir etwas gekauft zu haben. Die Qualität war mittelmäßig 
aber nach einem Joints hat man dann doch ordentlich was gemerkt. Blandine wurde langsam kalt und sie ist in ihr 
Dormitory. Dort hat sie Masha kennen gelernt, eine Russin die seit 3 Jahren in Deutschland lebt. Wir hatten also 
für den nächsten Tag eine sehr internationale Truppe zusammen um den zweiten Teil des Treks zu absolvieren. 
Der zweite Tag war nicht so anstrengend wie der erste, aber von dem Szenerie noch einmal beeindruckender. Wir 
sind die Schlucht hinabgestiegen und waren direkt am Yangtse, der sich an dieser Stelle zu einem wirklich reißenden 
Fluss entwickelt hat. Über eine kleine Holzhängebrücke konnte man auf einen Felsen inmitten der Strömung gehen.
Ich musste mich etwas überwinden, aber es war wirklich beeindruckend inmitten des Flusses und der Schlucht zu stehen.
Es ist wirklich schwer zu beschreiben, wie beeindruckend die Tiger Leaping Gorge ist. Wenn der Blick die Bergwand 
nach oben wandert, oder die Schlucht entlang schaut, oder auf den Yangtse hinabschaut. Alles lässt einen kurz innehalten 
und dann ein großes Glück und eine große Ehrfurcht vor der Natur fühlen. Wirklich ein absolutes Highlight!

22. September 2018, Flower Theme Hotel, Lijiang

Heute war unser Plan zum \emph{Jade Dragon Snow Mountain} zu gehen. Das ist ein Bergmassiv mit 13 Gipfeln. Es gibt 
dort mehrere Möglichkeiten diesen Nationalpark zu erleben, alles zu verschiedenen Preisen natürlich. Da es sehr bewölkt 
war, haben wir uns gegen die teuerste Variante, den Glacier Park, entschieden. Es hätte so nämlich keine schöne 
Aussicht gegeben. Aber erstmal hieß es dort anzukommen. Wir hatten uns am Vorabend den Ort heraus gesucht, an dem die 
Busse dort hin fahren. Als wir früh morgens aber dort waren konnte uns keiner helfen Tickets zu kaufen. So haben wir 
uns mit der Hilfe eines Chinesen umentschieden. Es gab eine Busverbindung im öffentlichen Verkehrsnetz die uns nur 
2 Yuan anstelle der 40 Yuan kosten würde. Der Nachteil wären 5km Fußweg am Ende. Das war für uns beide aber kein 
Problem und so haben wir uns für die billigere Variante entschieden. Das hat alles auch recht gut geklappt. Was 
wir nicht wussten waren die 7km nach dem Eingang zum Nationalpark bis zu dem Ort an dem wir weiterführende Tickets 
kaufen konnten. In China sind viele Sehenswürdigkeit so aufgebaut. An einem Ort bezahlt man den Eintritt und dann muss 
man noch mehrere Kilometer weiter zum eigentlichen Eingang. Es gibt dann natürlich immer Busse oder Battery Cars, 
aber die sind nicht kostenlos. Im Falle des Jade Dragon Snow Mountain Nationalparks mussten wir noch diese 7km 
überwinden bevor wir einen Bus bezahlen mussten um zu den interessanten Orten zu gelangen. Wir haben uns also auf den 
Weg gemacht. Blandine hatte dann aber eine super Idee, nämlich 10 Yuan raus zu halten, per Anhalter fahren mit einer 
kleiner monetären Gegenleistung. So hat es nur wenige Minuten gedauert bis uns jemand mitgenommen hat. Am Ziel 
angekommen mussten wir uns dann entscheiden, welchen Ort des Parks wir sehen wollten. Wir haben uns für die günstigste 
Cable Car Variante nach \emph{Spruce Meadows} entschieden. Die Landschaft war nicht wirklich überwältigend und es
gab nur einen kurzen Rundweg auf einem kürzlich angelegtem Weg bestehend aus Holzplanken. Die Luft war trotzdem 
angenehm und bei einem kurzen Abstecher abseits des Weges haben wir auch ein Yak erblickt, welches uns beiden sofort
Respekt einflösste. Nach den Spruce Meadows haben wir uns noch das \emph{Blue Moon Valley} angeschaut. Eine relativ 
schöne Aneinanderreihung aus mehreren Seen und Wasserfällen. Ich bin mir allerdings nicht sicher ob dieses 
künstlich angelegt ist oder nicht.
Danach gab es schon nicht mehr viel mehr zu tun ohne noch mehr Geld auszugeben, also wollten wir uns auf den Rückweg 
machen. Wir waren uns noch nicht ganz sicher, wie wir das bewerkstelligen würden, als und ein Fahrer eines kleines 
Transporters fragte ob wir für 25 Yuan nach Lijiang fahren wollten. Nach kurzer Überlegung haben wir uns dafür 
entschieden, auch weil es schon anfing zu regnen. Was wir nicht wussten war, das wir noch fast zwei Stunden warten 
würden bevor es endlich losgeht. Der Fahrer wollte natürlich den ganzen Transporter voll kriegen. Also hieß es für 
uns warten. Das war etwas nervig, aber da es drauen kräftig regnetete auch kein Weltuntergang. Trotzdem waren wir 
zum Ende schon etwas genervt. Aber schlussendlich hat alles gepasst und wir sind zurück nach Lijiang gefahren.

24. September 2018, Mrs Panda Hostel, Chengdu

Heute waren wir bei den Pandas! Und sie haben gehalten, was wir uns versprochen haben. Pandas sind wirklich 
unglaublich coole Tiere. Wir waren pünktlich um 7:30 morgens vor den Toren der Chengdu Research Base und es 
gab schon eine relativ große Schlange. Aber wir hatten viele Menschen erwartet, immerhin kann man hier Pandas sehen.
Die Anlage ist sehr schön gestaltet. Auf relativ großem Gebiet gibt es mehrere natürlich gehaltene Gehege für 
die Pandas. Außerdem gibt es noch zwei Gebäude wo sich um die Panda-Babies gekümmert wird. Wir sind erst einmal 
zu einem Gehege für jüngere Pandas gegangen, zwischen einem und zwei Jahre alt. Kurz nachdem wir angekommen sind, 
wurden sie in das Freigehege gelassen wo der frische Bambus schon auf sie gewartet hat. Was für ein Anblick diesen 
Tieren beim Essen zuzusehen! Ich habe noch nie chilligere Tiere gesehen. Sie sitzen in möglichst bequemer Position 
umringt von Bambus und schnappen sich ein Stück nach dem anderen und essen diese auf. Die äußerste Schale des Bambus 
essen sie nicht und es ist erstaunlich wie sie es schaffen beim abbeißen eines Stückes die Schale des nächsten 
Stückes schon zu entfernen. So sithen die Pandas ganz entspannt und essen vor sich hin. Ein witziges Highlight 
war ein schon älterer Panda der eine kurze Pause vom Essen eingelegt hat.Er hat sich kurz auf dem Bauch gelegt, 
den Besuchern sein Hinterteil gezeigt und eine große grüne Wurst raus gedrückt. Als er fertig war hat er sich wieder 
umgedreht, in beiden Pfoten hatte er schon Bambus um direkt mit dem Essen weiter zu machen. Pandas in Gefangenschaft 
essen pro Tag ungefähr 12 Stunden, und die anderen 12 Stunden schlafen sie um den Bambus zu verdauen. Dies ist gar 
nicht so ertragreich und einen Großteil müssen sie wieder ausscheiden, daher werden von den 20 - 40 kg Bambus pro Tag
ungefähr die Hälfte wieder in der Natur verteilt. Die Ruhepausen zwischen den Mahlzeiten haben sie auch perfektioniert. 
In möglichst bequemer Position entweder auf dem Boden, auf einem der Holzkonstruktionen oder oben in den Bäumen 
chillen sie bei gelegentlichen Positionswechseln. Denn jeder weiß keine Position ist dauerhaft wirklich bequem. 
So putzig diese Tiere sind, umso mehr musste ich daran denken, dass der Mensch mit seiner Gier nach Ackerland die 
natürlichen Wohnräume der Pandas extrem demeziert hat. Es gibt nur noch 1864 Pandas in freier Wildbahn! Das Projekt 
in Chengdu trägt dazu bei, dass Pandas wahrscheinlich nicht aussterben werden, Menschen finden sie zu niedlichen um das 
zu zulassen. Aber wenn ich heute die Menschen beobachtet habe, beschleicht mich das Gefühl, dass viele lieber Pandas 
in Zoos als in der Natur sehen würden. Leute klatschen und rufen den Pandas zu um deren Aufmerksamkeit zu kriegen, 
obwohl diese offensichtlich gerade schlafen (wollen). Das hat mich wirklich geärgert, was ich einer klatschenden 
Frau auch deutlich gezeigt habe. Ansonsten bleibt nur anzumerken, dass Pandas all das sind was man von ihnen erwartet, 
süß und putzig, nur am essen und schlafen. Pandas schaffen es wie wohl keine anderen Tiere auf diesem Planeten 
den Überlebenskampf tiefenentspannt darzustellen.

26. September 2018, Teddy Bear Hostel, Baoguo (Emeishan)

Gestern sind wir von Chengdu mit dem Zug nach Emei gefahren. Die Zugfahrt war etwas enger als sonst, das Abteil 
war komplett voll und dann wollten wir beide noch mit unseren großen Rucksäcken hinein. Das hat natürlich alle 
Blicke auf uns gezogen, noch mehr als normalerweise. Aber mit dem Rucksack zwischen den Beinen für zwei Stunden 
hat sich als machbar herausgestellt. So sind wir abends in Emei angekommen und mussten noch irgendwie zu unserem 
Hostel kommen, ungefähr 10 km von dem Bahnhof entfernt. Erst einmal mussten wir uns an den ganzen Leuten vorbei 
drängeln, die am Bahnhofsausgang schon warteten um uns ihre Fahrdienste anzubieten. Wir wollten natürlich den Bus 
nehmen, waren uns aber nicht ganz sicher wo sich die Haltestelle befindet, die wir brauchen. Bei unserem \emph{best guess} 
haben wir eine der Frauen gefragt, aber so richtig konnte sie uns nicht helfen. Aber einer andere Frau wurde auf 
uns aufmerksam und hat uns angeboten uns zu unserem Hostel zu fahren, wir müssten nur warten bis ihr Sohn von der 
Schule zurück kommt und dann würde sie uns fahren. Was für ein unfassbar nettes Angebot. Wir haben also auf den 
Schulbus ihres Sohnes gewartet. Es fuhren drei anderen Schulbusse vor, in denen die Grundschüler sofort erkannt haben, 
das zwei Ausländer an der Bushaltestelle stehen. So sind alle sofort zu den Fenstern gekommen um uns zu zuwinken. 
Ein wirklich süßes Schauspiel. Im vierten Bus war dann der Sohn unserer hilfsbereiten Chinesin. Diese hat uns dann 
wirklich bis vor die Haustür des Teddy Bear Hostels gefahren. So sind wir kostenfrei und sehr schnell an unser Ziel gekommen.
Das Hostel machte auch einen wunderbaren Eindruck, besondern das Bad und die Dusche haben uns überzeugt. Ein schön 
großes Bett gab es auch, was uns später noch sehr nützlich war.
Heute hatten wir für den ganzen Tag geplant den Buddha von Leshan, ungefähr eine Stunde mit dem Bus entfernt, 
zu bewundern. Der größte sitzende Buddha dieser Welt, über 70m hoch, vor 1200 Jahren in den Fels gehauen, wo sich 
drei Flüsse zu einem vereinigen. Dort angekommen gab es trotz leichten Regens viele viele Leute die das gleich 
vorhatten wie wir. Nicht weiter überraschend. Aber der Anblick dieses Symbols menschlicher Zielsetzung war es wert. 
Es ist wirklich beeindrucked zu sehen, mit wie viel Liebe zum Detail die Menschen vor über 1000 Jahren den Buddha 
mit einfachsten Mitteln in die Felswand geschlagen haben. So haben Blandine und ich den ganzen Tag in dem Areal 
verbracht, der neben dem Buddha noch mehrere Tempel und Felsgräber beinhaltet. Für morgen hatten wir eigentlich 
geplant auf dem Emeishan zu steigen, einen der vier heiligen Berge des Buddhismus. Aber da ich leicht angeschlagen 
bin und das Wetter auch eher regnerisch sein soll, haben wir uns für dafür entschienden einen Entspannungstag im 
Hostel einzulegen.

30. September 2018, Han Tang House, Xi'an

Heute ist unser zweiter Tag in Xi'an. Wir sind vorgestern kurz vor Mitternacht per Schnellzug angekommen. Wir 
hatten im Vorfeld Sorge, weil wir nicht wussten, wie wir vom Bahnhof zum Hostel kommen. Wir befürchteten schon
ein Taxi nehmen zu müssen. Aber entgegen unserer Informationen fuhr noch die U-Bahn und so kamen wir sehr entspannt 
und günstig zu unserem Hostel. Gestern haben wir dann etwas die Stadt erkundet. In Xi'an gibt es einen großen 
Anteil an Muslimen, sehr untyptisch für China. Ich vermute es liegt daran, dass Xi'an der Anfang der Seidenstraße 
war. Einer der großen Sehenswürdigkeiten ist daher auch dass \emph{muslim quarter}. Mehrere Straßen neben einer 
1300 Jahre alten Mosche, die eine Mischung aus chinesischem Markt und arabischem Basar sind. Überall wird Fleisch 
von den Rippen der Rinder geschnitten und scharf angebraten auf Holzspießen angeboten. Das Essen ist dort überall 
vorzüglich und man kann ohne Probleme 1-2 Stunden durch die Straßen schlendern und das Treiben beobachten. Zwei 
weitere Sehenswürdigkeiten der Stadt sind der \emph{Bell Tower} und der \emph{Drum Tower}, zwei große Gebäude 
aus der Zeit der Ming Dynasty, so gegen 1350. Diese wurden aufwendig restauriert und erstrahlen nun im alten Glanz.
Besonders in der Nacht ist das wirklich ein grandioser Anblick.
Heute haben wir uns vorgenommen die Terrakotta Armee zu besuchen. Eine der Sehenswürdigkeiten, auf die ich mich 
besondern gefreut habe. Da morgen der Beginn der \emph{Golden Week} ist, eine Woche lang Feiertag für die ganze Nation,
haben wir uns besondern früh auf den Weg gemacht um dem Besucheransturm zuvor zu kommen. Unsere Befürchtungen haben 
sich aber nicht bewahrheitet. Es gab schon viele Besucher, aber bei weitem nicht so viele wie ich gedacht hätte.
Wahrscheinlich gibt es morgen das große Gedränge. Wir haben gelesen, dass am 1. Oktober 2012 460000 Menschen die 
Terrakotta Armee besucht haben. Das muss kein Spaß gemacht haben! Wir hatten aber wie gesagt Glück und haben uns 
die Armee von Qin, dem ersten Kaiser Chinas, angeschaut. Es ist wirklich beeindruckend mit wie viel Liebe zum Detail 
diese Tonkrieger vor über 2000 Jahren gefertigt wurden. An vielen Stellen wird aber noch ausgegraben und restauriert. 
Von den über 6000 vermuteten Tonkriegern wurden erst ungefähr 1000 wieder aufgebaut. Ich hatte erwartet, dass man 
an die Hauptausgrabungsstelle kommt und Krieger sieht soweit das Auge reicht. Diese Erwartung wurde aber nicht 
erfüllt. Nichtsdestotrotz war es sehr interessant und beeindruckend zu sehen, welcher Aufwand auf Befehl des 
Kaisers betrieben wurde, nur damit er sich nach dem Tod gut beschützt fühlt. Jetzt sind wir beide wieder im Hostel 
angekommen und ruhen uns etwas aus. Wir sind beide etwas angeschlagen und husten vor uns hin. Heute Abend wollen wir 
aber noch eine Runde auf der historischen Altstadtmauer von Xi'an machen. 

3. Oktober 2018, Han Tang House, Xi'an

Die ganze Runde auf der historischen Stadtmauer haben wir nicht geschafft. Es war schon recht spät und die Mauer 
ist länger als wir gedacht hatten. Nach mehr als 90 Minuten hatten wir gerade mal die Hälfte geschafft und so haben 
wir uns entschlossen vom North Gate aus was zu essen zu suchen und dann zurück zum Hostel zu gehen. Immerhin mussten 
wir am nächsten Tag fit sein, weil wir uns vorgenommen haben den Huashan zu besteigen, oder besser gesagt vier seiner 
fünf Gipfel.
Wir dachten es ist eine gute Idee dem Golden Week Trubel der Stadt zu entfliehen und auf dem zwei Stunden entfernten 
Huashan zu wandern. Natürlich war das etwas kurz gedacht, da dieser Berg einer der Hauptsehenswürdigkeiten der Region 
ist. Es war zwar viel los, aber wahrscheinlich deutlich weniger als in Xi'an. Wir hatten gehofft es würde eine schöne 
Wanderung werden, allerdings haben wir nicht erwartet, dass alle Wege \emph{paved walks} waren. So war es eher 
ein langer und anstrengender Spaziergang als ein wirklicher Trek. Nichtsdestotrotz war es trotzdem sehr schön. 
Wir mussten mehrere Tausend Treppenstufen überwinden um bis zum ersten Gipfel, dem North Peak zu gelangen. Insgesamt 
gibt es fünf Gipfel, North, South, East, West and Central Peak. Unser Plan war es den Central Peak zu umrunden und 
dabei die anderen vier Gipfel zu besteigen. Die Aussicht während dieser Rundtour war atemberaubend, wenn auch nicht 
immer verfügbar. Häufig ist man lange im Schatten der Bergwände und Bäume gelaufen, bis man dann erst in Gipfelnähe 
die Aussicht genießen konnte. Dabei war es erstaunlich kühl, was wir so nicht erwartet haben. Trotzdem haben 
wir diesen Tagesausflug wirklich genossen. Die Natur spricht uns beide wirklich am meisten an. Die Herrscharen an 
Chinesen die sich am gleichen Tag auch auf den Weg gemacht haben waren aber sichtlich mehr an den mit chinesischen 
Schriftzeichen behauenen Steinen interessiert, vor denen sie eine Vielzahl an Selfies gemacht haben. Witzig 
zu beobachten war auch, wie sich viele Chinesen den Berg hochquälen mussten. Der Weg ist keineswegs einfach. Die 
Vielzahl an Treppen ermüden jeden und teilweise sind die Treppen wirklich steil. An dem extremsten Ort gab es 80\% 
Steigung. Trotzdem war es lustig und auch etwas traurig eine Vielzahl an Leuten in unserem Alter auf allen vieren 
die Treppen hochkrabbeln zu sehen. Sicherlich sollten sie mehr Ausdauer haben. Besonders weil das keine Auffälligkeit 
am Ende war, sondern schon am Anfang des Anstiegs. Naja, sowas passiert wohl, wenn Berge zu Orte des Massentourismus 
gemacht werden. Und bei der Landschaft dieses Bergmassivs wundert es nicht, dass es einer der 5 heiligen Berge Chinas ist.
Tiefe Schluchten, steilge glatte Felswände und dicht bewachsene Berghänge prägen die Umgebung und entlohnen für 
die Mühe des Aufstiegen. 
Nach unserem Rundgang der etwas länger als acht Stunden, über 40000 Schritte und wahrscheinlich 20000 Treppenstufen 
umfasste sind wir müde aber glücklich mit dem Bus zurück nach Xi'an gefahren. Die nächsten Tage haben wir uns 
vorgenommen nichts zu machen und einfach zu entspannen. Das passt auch ganz gut, weil durch die Golden Week 
die ganze Stadt überlaufen ist und man sich nur sehr schwer entspannt von A nach B bewegen kann.

5. Oktober 2018, Green Island Youth Hostel, Datong

Puh, das war wirklich eine schwere Ankunft in Datong. Die Anfahrt mit dem Nachtzug war noch sehr gut. Die Betten 
waren sauberer als ich erwartet hatte und die Fahrt hat sich auch nicht nach 16 Stunden angefühlt. Aber dann sind 
wir in Datong angekommen ...
Wir haben über Agoda ein Zimmer im Romantic Theme Guest House gebucht. Aus dem Zug ausgestiegen haben wir uns auf die
Suche gemacht. Dann plötzlich ist unser Booking als \emph{cancelled} in der App angezeigt worden. Blandine ist 
sofort etwas panisch geworden und sie war mit der Situation etwas überfordert. Ich habe dann erst einmal gesagt, 
wir müssen das Hotel finden um dann zu klären, warum es gecancelled wurde. Aber es war gar nicht so einfach zu 
finden. Auf maps.me und Agoda wurden zwei verschiedene Orte angegeben und weder an dem einen, noch an dem anderen 
haben wir das Hotel gefunden. Die Nerven bei Blandine lagen schon blank. Ich habe dann in einem China Post Gebäude 
einer der Mitarbeiterinnen gefragt ob sie uns helfen könne. Sie hat uns dann auch netterweise zu der Adresse gebracht, 
die für das Hotel angegeben war. Und dann mussten wir feststellen, dass das Romantic Theme Guest House schon seit
einiger Zeit geschlossen ist. Danke Agoda! So eine Scheiße. Was konnten wir jetzt tun? Durch die Golden Week gab 
es nur noch Hotelzimmer für über 50 Euro die Nacht, etwas über unserem Budget. Wir haben dann ein anderes Hostel
in der gleichen Straße gesehen, das Green Island Youth Hostel. Wir sind dort erst einmal reingegangen um zu schauen 
ob wir Glück haben und sie vielleicht noch zwei Betten für uns haben. Leider gab es nur eins, aber sie haben uns 
angeboten mit ihrem WiFi nach einer Alternative zu suchen. Während wir das getan haben, hat ein anderer Gast seine 
Reservierung storniert und wir hatten nun doch für die kommenden zwei Nächte einen Platz zum schlafen. Da haben 
wir wirklich noch einmal Glück gehabt. Aber am Ende findet sich wirklich immer eine Lösung!
Den Abend sind wir noch in die Altstadt gegangen um was zu essen. Das war aber eher enttäuschend. Die Stadtmauer 
sieht sehr schön aus aber innerhalb der Mauern wirkt es eher wie eine große Baustelle und wirklich nichts ist es 
wert sich anzuschauen. Wir haben erst später gelesen, dass ein früherer Bürgermeister von Datong sich zum Ziel 
gesetzt hat die ganze Altstadt neu und schicker wieder aufzubauen. Nur während des Großprojekts wurde er versetzt 
und seitdem gehen die Arbeiten wenn überhaupt nur sehr langsam voran. Clever ist das nicht gerade, aber nun ja.
Heute sind wir zu den Yungang Caves gefahren, eine der beiden Sehenswürdigkeiten in der Umgebung von Datong. Das 
sind mehrere Höhlen, die in den Fels gehauen wurden und in denen verschiedene Buddhastatuen zu sehen sind. Das war
recht schön anzusehen, aber auch nicht mega spektakulär. Den Rest des Tages entspannen wir jetzt noch und morgen früh 
geht es auf nach Peking, unsere letzte Station in China.

8. Oktober 2018, Leo Hostel, Peking

Heute ist mein 29. Geburtstag und Blandine hat dafür gesorgt, dass ich einen sehr schönen Tag hatte. Nachdem wir etwas 
verspätet vom Hostel aufgebrochen sind, wir haben einem älteren Briten ein paar Reisetipps gegeben, haben wir uns auf 
Weg zum 798 Art District gemacht. Eine sehr ruhige entspannte Gegend mit vielen hippen Cafés und Restaurants und 
natürlich vielen Kunstgallerien und viel Street Art. Das hat mir wirklich sehr gefallen, auch wenn leider viele 
Gallerien geschlossen waren. Am Nachmittag sind wir dann zum \emph{Red Theater} gefahren um uns Tickets für die 
abendliche Kung Fu Show anzuschauen. Es war Blandines Idee mich für meinen Geburtstag auf diese Show einzuladen.
Die Tickets zu kaufen hat sich als etwas schwierig heraus gestellt, aber nachdem das geschafft sind wir zum nahe 
gelegenen Park des Tempels des Himmels gegangen, einer der Hauptatraktionen Pekings. Der Park war der bisher schönste 
in China, was sicherlich an den beeindruckenden Gebäuden liegt. Und da die Golden Week zu Ende ist, gab es auch nicht 
allzu viele Leute, so konnten wir entspannt den Park erkunden. Um 19:30 begann dann die Kung Fu Show. Es war natürlich 
kein klassischen Kung Fu, sondern eher eine Mischung aus Artistik, Kung Fu und Ballett (in der Reihenfolge), aber 
trotzdem eine wirklich beeindruckende Show mit einer schönen, wenn auch einfachen Geschichte. 
Die ersten beiden Tage in Peking waren eher ruhiger. Wir sind vor zwei Tagen gut mit dem Zug angekommen, auch wenn 
wir von den sechs Stunden Fahrzeit ungefähr fünf stehen mussten. Das war anstrengend aber machbar. Die Ankunftszeit 
haben wir mit gefühlt mehreren Tausenden Chinesen geteilt. Es gab so unfassbar viele Menschen in dem Bahnhof, die alle 
nach der Golden Week zurück nach Peking gefahren sind. Wir sind dann zu Shi Shu gefahren, ein chinesisches Mädel, 
welches ich während meiner Zeit in Michigan kennen gelernt habe. Sie war für einen Monat in Ann Arbor und hat mir 
damals schon gesagt, dass ich mich auf jeden Fall bei ihr melden soll wenn ich während meine Weltreise in Peking bin.
Netterweise können wir auch bei ihr zwei Nächte schlafen, unsere erste und letzte Nacht in Peking. Sie hat sich 
wirklich alle Mühe gegeben eine gute Gastgeberin zu sein und uns sehr gut bekocht. Alles hat sehr gut geschmeckt, 
bis auf den Nachtisch. Das war ein Fungus mit Lotuskernen. Der Geschmack war wirklich sehr gewöhnungsbedürftig 
und meinem Magen hat es auch nicht bekommen, so dass ich mich während der Nacht übergeben musste. Glücklicherweise 
hat Shi Shu davon nichts mitbekommen. 
Am Tag darauf sind wir dann in unserem Hostel eingecheckt und haben am Nachmittag noch etwas Peking erkundet. 
Genauer gesagt sind wir zum Lama Tempel gefahren, ein sehr schöner tibetanisch-buddhistischer Tempel der als der 
schönste Pekings gilt. Und diese Aussage wundert mich nicht. Der Tempel ist wirklich sehr schön, auch wenn die 
Mengen an Räucherstäbchen etwas anstrengend sind. Nach dem Tempel sind wir durch die \emph{Hutong} geschlendert, 
einem Stadtviertel das dem historischen Peking entspricht. 

13. Oktober 2018, Beijing International Airport

Ereignisreiche Tage liegen hinter uns und das Kapitel China ist beinahe abgeschlossen. Wir sitzen am Gate 9 des 
Terminals 2 und warten auf unseren Flug der in 90 Minuten Richtung Hanoi aufbrechen wird. Wir haben China mit einem 
wirklichen Highlight abgeschlossen, der Großen Mauer von China! Dafür sind wir vorgestern nach Jinshanling gefahren, 
einem kleinen Ort 140km entfernt von Peking, deren einzige Existenzgrundlage Tourismus zur Großen Mauer ist. Viele 
Touristen besuchen die Mauer während einer Tagestour von Peking aus. Allerdings sind diese Touren natürlich relativ 
teuer und häufig gibt es an diesen Orten auch sehr viele Touristen. Wir wollten hingegen versuchen die Mauer zu erfahren 
ohne umringt zu sein von lärmenden Touristen. Wir sind also gegen 11 Uhr morgen nach Jinshanling aufgebrochen und dort 
problemlos zwei Stunden später angekommen. Dann mussten wir uns noch eine Übernachtungsmöglichkeit suchen, die wir 
relativ schnell für 120 Yuan gefunden haben. Allerdings mussten wir diesen Preis nach ursprünglich vorgeschlagenen 
220 Yuan noch raus handeln. Am nächsten Morgen sind wir dann schon 4:30 aus dem Bett gekrochen. Blandine wollte 
unbedingt den Sonnenaufgang von der Mauer aus sehen, seine sehr gute Idee!. Wir haben uns also möglichst viele 
Sachen angezogen, da es draußen Temperaturen um den Gefrierpunkt gab. Die Mauer haben wir dann schnell erreicht und 
wir hatten noch Zeit uns einen passenden Wachturm zu suchen um einen möglichst schönen Ausblick zu haben. Und was 
für einen Ausblick wir hatten! Die Sonne stieg lansam über die weit entfernten Berggipfel empor und das Ausmaß der
Großen Mauer von China hat sich langsam vor unseren Augen angedeutet. Entlang des Grates schlängelt sich dieses 
Bauwerk entlang so weit das Auge blicken kann. Wachtürme sind alle 50 oder 100 m zu sehen und man kann nicht anders 
als beeindruckt zu sein von dieser Konstruktion! Da wir so früh da waren hatten wir quasi die Mauer für uns alleine 
bis ungefähr 10 Uhr die ersten anderen Touristen angekommen sind. Dieses Alleinesein verstärkt noch die Mgie, die 
man spürt während man die Mauer entlang geht. Bis zum Mittag haben wir uns die Zeit damit vertrieben die Mauer 
soweit man gehen konnte zu erkunden und dann sind wir zurück nach Beijing gefahren.
Die Tage zuvor haben wir uns neben dem Lama Tempel weitere Attraktionen von Peking angeschaut. Den \emph{Temple 
of heaven}, den \emph{Summer Palace} und natürlich die Verbotene Stadt! Die Verbotene Stadt ist auch eine Sehenswürdigkeit 
die man so schnell nicht vergessen wird. Eine sehr große Anlage einzig zu dem Zweck gebaut ein zu Hause für die kaiserliche 
Familie zu sein. Alles ist nach den Prinzipien des \emph{Feng Shui} angelegt. So wechseln sich auf der Hauptachse 
große Paläste und Tore ab. Die Paläste entsprechen allerdings nicht dem, was sich Europäer darunter vorstellen 
würden. Es sind eher kunstvoll verzierte große Gebäude mit einem oder wenigen Räumen die einen bestimmten Zweck 
erfüllten, Ankleide des Kaisers, Ort für politische Diskussionen, etc. Für was welche Räumlichkeiten genutzt wurden 
haben die Kaiser aber immer für sich entschieden. Östlich und westlich dieser Hauptachse befanden sich die Wohnorte 
des ganzen Hofstabes und der Kaiserin, sowie die der Konkubienen. Die ganze Anlage ist sehr beeindruckend uns strahlt 
zweifelsohne einen Herrschaftsanspruch aus. Wirklich gemütlich wirkte alles aber nicht wirklich, bis auf vielleicht 
der kleine Garten ganz im Norden der verbotenen Stadt.


Die erste Etappe unserer Weltreise ist zu Ende. China, ein Land mit wunderschöner Natur und wachsenden Mega Cities.
Es war ein Abenteuer aber Blandine und ich haben es sehr gut gemeistert und konnten das meiste aus unseren 40 
Tagen im Land heraus holen. Wir sind wirklich sehr zufrieden mit den Erfahrungen die wir gemacht haben! Aber 6 
Wochen reichen wirklich. Wir haben schon in den letzten Tagen gemerkt, dass wir etwas müde wurden uns in China 
von A nach B zu bewegen. Es ist alles möglich, aber es erfordert eben auch etwas mehr Aufwand. Niemand spricht 
Englisch, es gibt fast immer ein großes Gedränge und die ständigen Sicherheitskontrollen in Bahnhöfen und U-Bahnstationen
sind auch einfach lästig. Trotzdem bleibt festzuhalten das Chinesen im Generellen wirklich hilfsbereit sind. An manche 
Dinge konnten wir uns aber auch nach all den Tagen einfach nicht gewöhnen. Beispielsweise das ständige Rumgespucke.
Jeder, ob alt oder jung, Mann oder Frau holt sich nach Belieben von ganz tief unten, unter Zuhilfenahme einer Vielzahl 
an Geräuschen, die Rotze hoch und spuckt sie auf den Gehweg, oder in die Bahnhofshalle, oder unter den Restauranttisch, 
oder aus dem Taxi raus. Auch der rücksichtslose Verkehr, gerade von Rechtsabbiegern, ist anstrengend. Im Gegensatz zu 
Vietnam wird nicht an den Personen vorbeigefahren sondern erwartet das man stehen bleibt, damit das Auto ungebremst 
um die Kurve ziehen kann. Das mir das überhaupt nicht gefällt, hat ein Taxifahrer in Datong, beziehungsweise seine 
Scheibe zu spüren bekommen. Ansonsten merkt man einfach, dass China sehr weit weg von Europa und der westlichen Welt 
ist und sich hier viele Besonderheiten herausgebildet haben die uns eher kurios erscheinen. Da sind beispielsweise 
die Hosen für kleine Kinder, die einen großen Schlitz zwischen den Beinen haben, sodass der enweder vorne oder hinten 
alles rausguckt. So ist es einfacher für die Eltern die Kinder mal fix neben dem Busch pinkeln zu lassen. Allerdings 
hat es den Anschein gehabt, dass die Eltern eher wollten, dass die Kinder auf den Gehweg pullern. Auch sehr kurios 
sind die Tanzveranstaltungen auf den großen Plätzen. Dort kann jeder der Lust auf Partnertanz hat hingehen und findet 
sofort jemanden des anderen Geschlechts um sich mit verschiedenen Tänzen die Zeit zu vertreiben. Das Komisch ist allerdings 
dass die Männer nie ihre Partnerin angeschaut haben. Eher haben sie desinteressiert weg geschaut, als ob sie von ihr 
gelangweilt waren. Vielleicht ist es aber auch einfach unhöflich der Dame in die Augen zu schauen. Oder natürlich 
die offline-Partnerbörse wo Eltern potentielle Partner für ihre Kinder finden. All das und noch vieles mehr wirkte 
sehr fremd für uns, aber trotzdem ist es spannend diese Kultur zu erleben!
Wir sind sehr froh China bereist zu haben. Viele unbeschreibliche Erfahrungen, wie der Tigerschluchtsprung, 
der Shilin Stone Forest oder die Pandas werden uns immer in Erinnerung bleiben und so verlassen wir China mit 
einem Lächeln auf dem Gesicht.

\chapter{Vietnam}
20. Oktober 2018, Ecopark in der Wohnung meinem Eltern
Seit schon einer Woche sind wir in Vietnam, dem Herkunftsland meiner Mutter. Blandine und ich waren wirklich müde 
von China und haben uns sehr auf Vietnam und insbesondere auf meinen Vater gefreut. Familie ist schon etwas besonderes.
Wir sind pünktlich in Hanoi Noi Bai angekommen. Mein Vater hat zwar nicht auf uns gewartet, da er wie üblich etwas 
zu spät war. Aber schon wenige Minuten nachdem wir in die Ankunfshalle gegangen sind kam er uns entgegen. Die Freude 
war groß und es fühlte sich sofort sehr heimisch an in Vietnam. Wir sind dann in Richtung Ecopark aufgebrochen
und haben dort ein deutsches Abendbrot gegessen. Richtiges Brot, Butter und Wurst. Das war wirklich sehr lecker 
und gerade Blandine war glaube ich sehr froh mal ein Abendbrot ohne Reis oder Nudeln zu essen.
Die nächsten beiden Tage haben wir mit meinem Vater in Hanoi verbracht. Zunächst sind wir in ein Dorf in der 
Nähe des Ecoparks gefahren, in dem meine Eltern immer auf dem Markt einkaufen. Es war cool zu sehen wie mein 
Vater auf Vietnamesisch mit dem Damen auf dem Markt geredet hat. Ich muss sagen, ich bin erstaunt, wie gut er doch 
Vietnamesisch reden kann. Wir haben also ein paar Dinge für das Mittag eingekauft und dann sind wir in die Altstadt gefahren. 
Mein Vater war etwas wie ein Touristenführer und er hatte viel zu erzählen zu verschiedenen Ecken der Stadt. Wir haben 
auch den Hoan Kiem See umrundet und uns natürlich die Schildkröte angeschaut die der Legende nach dem Kaiser erschien 
und ein Schwert auf dem Panzer trug. Diesen Abend haben Blandine und ich dann noch das Wasserpuppentheater besucht.
Ein Klassiker, den ich glaube ich schon drei oder viermal in Hanoi gesehen habe. Und immer wieder ist es ein Highlight,
besonders durch die traditionelle Live-Musik. Den nächsten Tag haben wir damit verbracht den Literaturtempel zu besuchen, 
einem meiner Lieblingsorte in Hanoi. Danach sind wir mit dem Moped einmal um Westsee gefahren, was auch sehr schön war.
Die nächsten drei Tage standen ganz im Zeichen der Ha Long Bucht. Wir sind am späten Vormittag mit dem Bus aufgebrochen 
und kamen am frühen Abend in Cat Ba Stadt an. Blandine wirkte erst etwas müde und erschöpft während der Busfahrt, 
aber als dann die ersten Teile der Ha Long Bucht zu sehen waren, war sie sofort in ihrem Element. Sie wirkte begeistert 
und ist hin und her gehüpft und ganz viele Fotos zu machen. Wie ich das liebe, wenn sie solch eine Freude ausstrahlt.
Am darauffolgenden Tag haben wir dann auch die Bootstour gemacht. Das Boot war gut gefüllt, aber zu keinem Zeitpunkt 
zu eng. Wir sind erst ungefähr zwei Stunden durch die Bucht gefahren um dann eine kurze Pause einzulegen, um mit 
Kajaks etwas durch die Höhlen und um die Inseln zu paddeln. Für Blandine war es das erste Mal in einem Kajak und nach
anfänglicher Nervosität hat sie sehr schnell Freude daran gefunden. Das Wetter war zwar nicht perfekt hatte aber 
auch seinen Charme, da die vielen Inseln oder besser gesagt Karstberge in der Ferne immer schemenhafter wurden, 
bevor sie verschwunden sind. Ich liebe diese Landschaft und könnte stundenlang auf einem Boot durch die Ha Long 
Bucht fahren und einfach in die Ferne schauen. Es ist immer wieder erstaunlich wie viele Inseln es sind, sieht es 
doch aus der Entfernung eher wie eine große Inseln mit sehr vielen Bergspitzen aus. Und doch ist quasi jeder Spitze 
eine eigene Insel. Ein weiter Stop war Monkey Island, die wie ihr Name schon verrät von vielen Affen, ungefähr wohl 
70, besiedelt ist. Dort konnten wir auf einen kleinen Berg klettern um eine schöne Aussicht zu genießen. Auf dem Rückweg
nach Cat Ba sind wir noch an einem Fischerdorf vorbe gefahren. Das Besondere ist, dass dieses Dorf komplett auf dem 
Wasser ist. Dort leben ungefähr 1000 Menschen auf schwimmenden Häusern. Zurück in Cat Ba haben wir den Tag entspannt 
bei einem Bier ausklingen lassen. Während wir das Bier trinken sieht Blandine eine Person die ihr bekannt vorkommt.
Es war David, ein Brite den wir in Peking in unserem Hostel kennen gelernt haben. Was für ein witziger Zufall. Er 
hat sich also zu uns gesetzt und wir konnten noch etwas erzählen. Solche Zufälle sind wirklich immer super! 
Am Tag darauf hatten wir noch den Vormittag in Cat Ba Stadt bevor unser Bus zurück nach Hanoi losfuhr. Das haben 
wir genutzt um noch einmal einen Strand aufzusuchen und zu baden. Ich habe natürlich nicht nur gebadet sondern endlich 
auch mal wieder die Möglichkeit gehabt etwas mein Kung Fu zu üben.
Der nächste Tag stand ganz im Zeichen der Familie. Zum Mittag wurden wir von Bac Viet und seiner Frau in ein vegetarisches 
Restaurant eingeladen. Zusätzlich kamen noch ein Cousin von Mutti mit seiner Frau und seiner Tocher. Das war eine sehr
angenehme Runde und ich konnte mich gut mit Bac Viet unterhalten. Ich wusste nicht, dass sein Englisch so gut war. 
Nach einem Kaffee sind Blandine und ich dann weiter zu Bac Duong gefahren. Dort wurden wir zum Abendessen eingeladen.
Mein Vater war nicht dabei, weil er sich nicht sehr gut mit Bac Duong versteht. Für uns war es trotzdem sehr schön, 
wir haben uns gut mit Bac Duong und seiner Frau Bac Van unterhalten und auch die kleine Tochter Lan Phuong hat etwas
erzählt, auch wenn sie sehr schüchtern ist. Alles in allem war es ein sehr schöner Tag und wir haben sehr gut gegessen,
wenn auch viel zu viel. Den Tag darauf haben wir dann alle etwas ruhiger angehen lassen. Ich war vormittags beim 
Friseur, weil die Matte endlich runter musste. Nachmittags haben wir dann Ongs Grab besucht und sind danach noch etwas 
durch die Stadt geschlendert. Dort haben Blandine und ich ein schönes Souvenir gekauft, eine reich verzierte Holzstatue.
Heute müssen wir uns dann von meinem Vater und Hanoi verabschieden. Wir brechen abends mit dem Nachtbus in Richtung 
Hue auf.

24. Oktober 2018, Hong Thien Hotel in Hue

Heute ist unsere letzte Nacht in Hue nachdem wir zwei sehr schöne und ereignisreiche Tage erlebt haben. Die Fahrt
von Hanoi nach Hue in dem Nachtbus war ganz in Ordnung. Das Problem war nur, dass die Liege ungefähr 10 Zentimeter
zu kurz für mich war und so konnte ich nicht so bequem schlafen. Aber trotzdem kamen wir voller Vorfreude pünktlich 
7 Uhr morgens in Hue an. Wir hatten auch Glück mit der Wahl unseres Hotels, da wir nur 50 Meter die Straße runter 
gehen mussten um dort anzukommen. Also haben wir nur schnell gefrühstückt und uns dann gleich ein Moped geliehen 
um unseren Tag in Hue zu starten. Wir sind zuerst zur Zitadelle, der verbotenen Stadt Vietnams, gefahren. Dort 
haben wir den ganzen Vormittag verbracht. Eine wirklich sehr schöne und ruhige Anlage. Im Gegensatz zur Verbotenen 
Stadt in Peking, die natürlich viel größer ist, hat die \emph{Imperial City} in Hue eine sehr viel wärmere Atmospähre.
Es gibt einen See und viele kleinere Gärten. Blandine hat es auch viel besser in Hue als in Peking gefahren. 
Ein Großteil der Gebäude ist zwar zerstört wurden, das tut der Magie dieser Sehenswürdigkeit aber keinen Abbruch.
Danach haben wir mit dem Moped noch drei Gräber von ehemaligen vietnamesischen Kaisern besucht. Das Moped fahren 
hat wirklich sehr viel Spass gemacht, auch wenn man natürlich gerade bei dem vietnamesischem Verkehr sehr vorsichtig 
sein muss. Die Gräber die wir uns angeschaut haben waren von den Kaisern Tu Duc, Khai Dinh und Ming Manh. Herausgestochen 
ist dabei definitiv die reich verzierte Grabanlage des Kaisers Khai Dinh. Die drei Gräber haben den ganzen Nachmittag 
in Anspruch genommen. Es war wirklich wunderschön diese mit dem Moped zu erkunden. Auf dem Rückweg in die Stadt 
haben wir noch einen Halt bei dem Literaturtempel Hues gemacht. Einer meiner Verwandten wurde dort geehrt, aber
ich wusste leider nicht auf welcher Steintafel. Den Abend haben wir dann bei einem schönen Abendbrot ausklingen lassen.
Für den heutigen Tag haben wir uns über das Hotel eine Tour in die DMZ gebucht, die demilitarisierte Zone ungefähr 
2 Autostunden nördlich von Hue. Die Tour hat mehrere Stops beinhaltet von denen der beeindruckendste definitiv die 
\emph{Vinh Moc Tunnel}. Dort wurde quasi ein Dorf in drei Stockwerken und bis zu 20 Meter unter der Erde errichtet, 
oder besser gesagt gegraben. Es ist das zweite Mal dass ich dort bin und es war wieder sehr beeindruckend. Die Enge 
der Gänge lässt uns nur erahnen wie schrecklich diese Erfahrungen für die damaligen Bewohner des Dorfes gewesen sein müssen.
Weniger schön war, dass wir in unseren ersten Unfall verwickelt waren. Unser Fahrer hat mit dem Van auf einer Kreuzung 
einen Zusammenprall mit einem Moped gehabt, auf dem zwei junge Mädels saßen. Es wurde erst sehr hektisch, aber glücklicherweise 
ist niemandem etwas passiert. Eine halbe Stunde wurde hin- und her diskutiert und etwas Geld hat den Besitzer gewechselt
und dann haben wir die Tour fortgesetzt. Wir müssen auf jeden Fall sehr vorsichtig sein, wenn wir uns morgen mit dem 
Moped Richtung Hoi An aufmachen.

1. November 2018, Minh Anh Garden Hotel in Mui Ne

So es ist passiert, der erste und hoffentliche letzte Krankenhausbesuch. Blandine ist am Strand wohl in eine Glasscherbe
getreten und hat sich einen recht großen Schnitt am Fuß zugezogen. Der Schnitt ist ungefähr 6 Centimeter lang und 
wurde mit 4 Stichen genäht. Seitdem wurden wir natürlich etwas ausgebremst, da sie sich viel ausruhen muss und nicht
gehen kann. Zusätzlich zu der Verletzung am Fuß kam dann auch noch, dass der Hund von ihrem Onkel Jean Francois sie 
an der Wade gebissen hat. Ja sie hatte wirklich kein Glück. Aber das wird sie auch alles überstehen. Wir haben 
insgesamt 4 Nächte in Hoi An bei ihrem Onkel verbracht. Das war das Glück im Unglück, denn so war alles etwas 
komfortabler die ersten Tage nach der Verletzung. Für mich war es auch sehr schön ein weiteres Familienmitglied von 
Blandine kennen zu lernen. Jean Francois war sehr nett und froh uns zu sehen, denn auch Blandine hatte er seit mehr 
als 7 Jahren nicht mehr gesehen. Er wohnt mit seiner vietnamesischen Frau Kim und deren Sohn Jet in einem schönen 
Haus. Vor der Verletzung konnten wir uns noch Hoi an anschauen, immer noch eine sehr schöne Stadt aber mittlerweile 
schon extrem touristisch. So geht leider etwas der Charme verloren. Von der Umgebung konnte Blandine dann leider 
nicht mehr sehen, weil sie sich natürlich größtenteils ausgeruht hat. Ich bin einen Tag nochmal mit dem Moped 
auf den Wolkenpass gefahren und zu einer Buddhastatue, der Lady Buddha. Das ist die größte Buddhastatue in Vietnam 
und war wirklich sehr schön und beeindruckend. 
Unsere Anreise nach Hoi An war auch sehr schön. Wir haben uns ein Moped gemietet mit dem wir von Hue über den Wolkenpass 
nach Hoi An gefahren sind. Die Strecke war wirklich sehr schön und Moped fahren macht einfach Spass. Wir waren auch 
an einem menschenverlassenen Strand baden und dort hat es Blandine noch sehr viel Spass gemacht in den großen 
Wellen hin und her zu springen. Auf dem Wolkenpass haben wir dann Jean Francois getroffen. Mit dem wir gemeinsam 
zu ihrem Haus gefahren sind.
Wir hatten geplant von Hoi An nach Mui Ne zu fahren und Blandine hat sich auch fit genug gefühlt die 18 Stunden 
Busfahrt auf sich zu nehmen. Das hat auch den Umständen entsprechend recht gut funktioniert und so sind wir vor 
2 Tagen gegen Mittag hier angekommen. Seitdem haben wir uns viel entspannt. Ich war zweimal am Strand um mein Kung 
Fu zu üben. Ich freue mich, dass ich den Strand gut nutzen kann um die Lau Gar Kuen zu wiederholen, da ich die 
letzten Wochen quasi gar nicht geübt habe. 

7. November 2018, Saigon Smile Hostel, Ho Chi Minh City

Heute ist unser letzter Tag in Vietnam. Morgen fahren wir mit dem Bus nach Phnom Penh in Kambodscha. Die letzten Tage in 
Mui Ne habe ich weiter genutzt um viel am Strand zu trainieren. Ich habe mir aber auch für einen Tag ein Moped ausgeliehen 
und habe die Sehenswürdigkeiten in der Umgebung erkundet, leider ohne Blandine. Ich bin zu den Sanddünen gefahren, die wie 
immer sehr cool sind. Der Sand ist heiß und die Sonne strahlt einen erbarmungslos auf den Kopf. Es ist faszinierend die Dünen 
hoch und runter zu laufen. Danach bin ich die Küste entlang gefahren um mir die Cham Türme anzuschauen. Die Cham waren eine 
Kultur die lange Zeit in Mittel- und Südvietnam vorherrschend war. Diese Türme sind interessant anzuschauen, aber nichts was 
einen vor Ehrfurcht erstarren lässt. Dann bin ich weiter zu einem Leuchtturm gefahren, den wohl größten in Südostasien. Das 
war eine ziemlich coole Szenerie. Der Leuchtturm ist auf einer felsigen Insel, die vielleicht 100 Meter von der Küste entfernt ist.
Vom Strand aus ergibt sich so eine wirklich schöne Perspektive auf den Leuchtturm und das dahinter liegende Meer. Danach bin 
ich wieder zurück zu Blandine gefahren. Das Highlight es Tages war es auf jeden Fall einfach mit dem Moped herum zu fahren. 
Ich bin verschiedene Schleichwege entlang gefahren, wo teilweise sehr loser Sand war. So bin ich quasi auf dem Sand hin und her 
geschwommen, während die Räder durchdrehten, während ich Gas gegeben habe. 
Vor drei Tagen sind wir dann mit einem Bus nach Saigon gefahren. Wir hatten Glück, dass unser Hostel direkt in der Nähe von 
dem Sinh Tourist Office ist, so konnte Blandine noch ohne Taxi zum Hostel gelangen. Wir hatten im Vorfeld Thomas, einen Bekannten 
meiner Eltern aus Wismar kontaktiert, damit dieser uns zum Krankenhaus begleiten konnte. Es war Zeit die Fäden zu ziehen, dachten 
wir zumindest. Der Arzt hat uns gesagt wir sollten nochmal in zwei Tagen zurück kommen, weil es bis dahin besser zusammen gewachsen 
sein sollte. So sind wir erst einmal zu Thomas nach Hause und haben dort noch Kaffee getrunken. Dann haben wir noch gemeinsam 
gegessen uns haben die Einladung von ihm und seiner Frau Ha angenommen ins Kino zu gehen. Wir haben uns das Freddy Mercury 
Biopic \emph{Bohemian Rhapsody} im Kino angeschaut. Und dieses Kino war das luxoriöseste in dem wir je waren. Wir saßen in Sesseln 
die man so verstellen konnte, dass man komplett lag. Wirklich sehr angenehm! Den zweiten Tag habe ich vormittags damit verbracht 
Krücken für Blandine zu kaufen. Das haben wir bisher vernachlässigt und so musste sie immer hüpfen um sich von A nach B zu bewegen. 
Das war gar nicht so einfach Krücken zu finden, aber am Ende habe ich welche gefunden. Damit fällt es Blandine jetzt deutlich einfacher
sich zu bewegen, auch wenn es natürlich immer noch anstrengend ist. Abends habe ich mich dann mit Quynh getroffen. Ich, oder besser 
gesagt meine Eltern, kennen ihn aus Wismar. Dort hat er einen Teil seines Studiums zum Anwalt absolviert. Das Treffen war sehr 
nett. Er hat eine eigene Anwaltskanzlei und wir sind mit seinen Angestellten und seiner Frau plus Tochter in ein Streetfood 
Restaurant gegangen, welches sich auf Seafood spezialisiert hat. Das Essen war in Ordnung, aber der größte Fan von Muscheln werde
ich wahrscheinlich nie werden. Nach dem Restaurant sind Quynh und ich noch in eine Skybar im Deutschen Haus gegangen. Die war sehr edel 
und sehr teuer, also war ich froh das er mich eingeladen hat. 

Vietnam, das Mutterland meiner Mutter und damit Hälfte meiner Identität liegt hinter uns. Die Erwartungen waren sehr hoch, von 
mir, aber auch von Blandine. Diese wurden eigentlich erfüllt, aber die zweite Hälfte der Reise stand natürlich auch im Schatten 
der Fußverletzung von Blandine. Wir haben viel erlebt, aber im Gegensatz zu China sind wir die Vietnam eher ruhiger angegangen.
Das lag unter anderem auch daran, dass wir Familie und Bekannte in Vietnam besucht haben, die unsere Planung etwas ausgebremst 
haben (nicht im negativen Sinn). Was mir von Vietnam bleibt ist vor allem das grandiose Essen. Ich könnte jeden Tag vietnamesisch 
essen. Banh Xeo, Pho, Cao Lau, Nem, Xoi, Banh Mi und vieles mehr ist einfach so gut! Auch die Landschaft ist grandios, insbesondere 
die Ha Long Bucht. Diese ist mittlerweile viel sauberer als in meiner Erinnerung. Auch das Moped fahren war immer wieder ein Highlight.
Es ist natürlich gefährlich auf den Straßen Vietnams, aber der Spaß ist da und wenn man vorausschauend fährt geht es auch. Ärgerlich 
bleibt weiterhin die Verschmutzung in der Stadt und auch an den Stränden und natürlich der rücksichtslose Verkehr. Das ständige 
plötzliche Bremsen während der Busfahrten lassen einen nicht nur einmal zusammen zucken. Auch bin ich etwas enttäuscht, wie sich 
der Tourismus in Vietnam nicht entwickelt. Damit meine ich nicht zwangsweise die Touristen die kommen, aber die Art und Weise 
wie die Vietnamesen mit dem Tourismus umgeht. Es wirkt vieles auf das schnelle Geld aus und nicht auf nachhaltigen Tourismus.
So hat meiner Meinung nach Mui Ne vieles von seinem Charme verloren. Auch Hoi An, was von eigentlich alles Touristen noch als 
Highlight ihrer Vietnamreise benannt wird hat vieles von seinem Zauber verloren. 
Trotz allem hat das Land nichts von seiner Anziehung auf mich verloren. Ab der ersten Sekunde habe ich mich wohl gefühlt. Ich bin 
in Vietnam immer ein Tourist gewesen und werde es wahrscheinlich auch immer sein, aber ich fühle mich nicht wie einer. 

\chapter{Kamboscha}

8. November 2018, Good Morning Guesthouse, Phnom Penh

Heute sind wir in der früh mit dem Bus von Saigon nach Phnom Penh gefahren. Die Busfahrt hat etwas mehr als 7 Stunden gedauert, 
wobei die Formalitäten an der Grenze recht lange gedauert haben. Blandine hat immer weniger Probleme mit ihrem Fuß. Das Ziehen der 
Fäden hat ihr viel Angst genommen glaube ich. Als wir in Phnom Penh angekommen sind haben wir ein Tuk Tuk zu unserem Hotel genommen 
und haben uns nach dem Mittagessen auf dem Weg zu einem schöne Tempel, Wat Ounalom, gegangen. Das war die erste Sehenswürdigkeit 
die Blandine seit ihrer Verletzung gesehen hat. Es war sehr schön zu sehen welche Freude sie hatte endlich mal wieder etwas zu entdecken.
Hoffentlich geht es jetzt mit ihrem Fuß immer besser! Ansonsten sind die ersten Eindrücke von Kambodscha sehr spannend. Das Land 
ist noch merkbar ärmer als Vietnam, auch wenn zumindest in Phnom Penh das Essen teurer als in Vietnam ist. Viele Kinder arbeiten 
auf der Straße und verkaufen alles mögliche von Essen, Luftballons oder Souvenirs.  


14. November 2018, Bird of Paradise Bungalows, Körper

Schon seit einigen Tagen entspannen wir uns in unserem Bungalow in Kep. Die Stadt, wobei man es eigentlich nicht Stadt oder sogar 
Dorf nennen kann, ist sehr ruhig. Auch wenn wir etwas Pech mit unserem Bungalow haben. Jeden Abend gibt es laute Musik vom nebenan 
oder gegenüber. Viele Chinesen genießen ihre Karaokeabende vermuten wir. Aber glücklicherweise ist es tagsüber sehr ruhig und abends 
geht es nur bis um 9. Also halb so schlimm. Der Bungalow ist an sich sehr schön aber der Service in der Anlage hier lässt sehr zu 
wünschen übrig. Nur die Hälfte der Gerichte auf der Karte sind verfügbar, Mopeds ausleihen ist an sich zwar möglich aber es sind 
immer alle schon weg, Fahrräder gibt es kostenlos, aber sie werden gerade repariert. Außerdem hatten wir gestern Abend eine große 
Kröte zu Besuch in unserem Badezimmer. Aber trotzdem fällt es uns ganz leicht die Zeit hier trotzdem zu genießen. Was auch daran liegt, 
das wir uns zum Mittag 1 Kilogramm frisch gefangene Shrimps gönnen. Abends haben wir uns auch schon Krabbe servieren lassen. Das 
schmeckt auch sehr gut, ist aber immer etwas Fummelei um wirklich alles gute zu bekommen. Shrimps sind da einfacher. Die kaufen wir 
einfach auf dem Markt und lassen sie dort auch gleich kochen. Dann werden sie von uns nur mit Limette, Salz und Pfeffer verspeist.
Gestern haben wir ein Moped ausgeliehen und sind nach Kampot gefahren, das ist ungefähr 25 Km von Kep entfernt. Dann sind wir weiter in den 
Bokor National Park, eine recht schöne Anlage mit sehr guten Straßen durch die wir wunderbar mit dem Moped düsen konnten. Das war 
wirklich super, auch wenn es oben am Berg doch recht kühl war. Nach dem Park sind wir zu \emph{La Plantation} gefahren, einer Pfefferplantage, 
die kostenlose Touren anbietet. Das war wirklich sehr cool, zu erfahren wie der Pfeffer angepflanzt und verarbeitet wird. Am Ende 
gab es natürlich noch eine Pfefferverkostung. Meine Zunge hat noch nicht die Fähigkeit entwickelt Pfeffer wirklich zu unterscheiden, 
aber trotzdem war es cool verschiedene Verarbeitungen zu probieren. Auch wenn es mir schwer gefallen ist die verschiedenen Sorten 
des Kampot-Pfeffers zu unterscheiden merkt man doch einen deutlichen Unterschied zu dem Pfeffer den wir in Deutschland kennen. Alles 
in allem war es ein sehr gelungener Tag und Blandine fällt es immer leichter zu gehen. Bald wird sie hoffentlich wieder voll fit sein.
Die Tage in Phnom Penh waren auch sehr schön bzw informativ. An unserem ersten vollen Tag haben wir uns ein Tuk Tuk gemietet. Der 
Fahrer hat uns zu den \emph{Killing Fields} außerhalb der Stadt gefahren. Hier wurden mehrere Tausend Menschen während des Regimes 
der Roten Khmer unter Führung Pol Pots getötet. Es ist dort nicht mehr viel zu sehen, da alle Häuser abgerissen wurden. Aber die 
Massengräber sind immer noch da und mit Hilfe des sehr guten Audio Guides hat mein ein Gefühl davon bekommen was für schreckliche 
Dinge dort passiert sind. Die Erfahrung dort ist sehr bedrückend, besonders wenn man erfährt das beispielsweise Babys vor den Augen 
ihrer Mütter gegen einen Baum geschlagen wurden. Die Mütter wurden kurz danach auch einfach umgebracht und in ein Massengrab geschmissen. 
Es ist schmerzhaft sich mit so etwas auseinander zu setzen aber auch wichtig. Es ist ein großer Teil der kambodschanischen Geschichte 
und damit auch Teil der menschlichen Geschichte! Nach den Killing Fields sind wir zu \emph{Tuol Sleng} oder auch S-21 genannt gefahren. 
Dem früheren Foltergefängnis der Roten Khmer in Phnom Penh welches nun das Genozidmuseum ist. Auch hier gab es einen exzellenten 
Audio Guide und man hat sich dieser schrecklichen Periode Kambodschas sehr nahe gefühlt weil man durch die Räume gegangen ist, in 
denen Menschen auf schlimmste Art und Weise gefoltert wurden. Tausende Menschen wurden gefoltert damit sie Geständnisse ihrer Schuld 
unterschreiben. So konnten sie "rechtmäßig" auf den Killing Fields getötet werden. Diese beiden Besichtigungen waren wirklich sehr 
intensiv und sie haben mich für die Menschheit schämen lassen. Wie kann so etwas nur möglich sein?
Den letzten Tag in Phnom Penh bin ich zum Royal Palace und der Silber Pagode gegangen. Eine schöne Sehenswürdigkeit mit reich verzierten 
Gebäuden und Tempeln. Außerdem war ich noch im Nationalmuseum von Kambodscha, in dem es viele Statuen zu sehen gab. Recht schön, aber 
meiner Meinung nach für 10 USD deutlich zu teuer. 


17. November 2018, Santepheap Guesthouse, Kampong Thomas

Heute war unser erster voller Tag in Kampong Thom. Geplant als kurzer Zwischenstop damit die Fahrt nach Siem Reap nicht zu lange ist, 
wurden wir heute wirklich überrascht mit dem bisher besten Tag in Kambodscha. Wir sind morgen mit dem Moped zu einem Tempel südlich 
der Stadt gefahren. Dieser ist auf einem kleinen Hügel von knapp 200m Höhe gelegen. Klingt nicht viel, aber die Stufen nach oben 
haben uns beide trotzdem schon ziemlich ins Schwitzen gebracht. Auffallend bei diesem Tempel waren all die Affen die überall am Klettern, 
Laufen und schlafen waren. Zwei Affen konnten wir auch bei etwas beobachten wofür sich Menschen typischerweise alleine im Schlafzimmer 
befinden. Aber bei diesen Affen war es eher die quick \& dirty Version. Nach dem Tempel waren wir noch kurz in einem Dorf am Fuße 
des Hügels. Dort gab es irgendeine große Festivität. Ich glaube es war eine Einweihung eines Tempels, auch wenn dieser noch im Rohbau 
war. Auf jeden Fall war viel los und diese ungestellte kambodschanische Atmosphäre war sehr interessant anzuschauen. Anschließend sind 
wir zu einer nahe gelegenen Seidefarm gefahren. Dort angekommen wurden wir auch gleich von der Eigentümerin begrüßt. Eine Laotin mit 
vietnamesischen Vater, die beide Eltern durch die Roten Khmer verloren hat. Sie hat uns voller Freude den gesamten Zyklus der Seidengewinnung 
gezeigt. Angefangen von den Seidenraupen, die sehr gefräßig sind, über die \emph{cocoons} hin zu den Schmetterlingen. Für die Seide 
werden nur die Kokone benötigt. Diese werden in kochendes beziehungsweise heißes Wasser gelegt und die Seidenfäden wird quasi davon 
abgrollt. Man braucht 40 Kokone um einen dünnen Seidenfaden zu produzieren. Das Gewichtsverhältnis zwischen Kokone und gewonnener Seide 
beträgt ungefähr eins zu zehn. Da wundert es nicht, dass Seide so teuer ist. Die kambodschanische Seide zeichnet sich durch ihre 
natürliche goldene Farbe aus. Das hat uns wirklich gut gefallen und wir haben uns als Souvenir einen Seidenschal gekauft. Das wirkliche 
Highlight des Tages kam aber im Anschluss. Ohne wirkliches Ziel sind wir den ganzen Nachmittag durch die Umgebung von Kampong Thom 
gefahren. Mehrere Stunden sind wir über buckelige Sandpisten gefahren. Links und rechts von uns waren intensiv grüne Felder von 
jungem Reis. Die Farbkombination von dem strahlend blauem Himmel, dem Grün der Reispflanzen und das staubige Rot der Piste waren wirklich 
wunderschön. Überall waren große Wasserbüffel bis auf den Kopf in dicken Schlammbädern versteckt und alle 100 Meter haben uns 
freudige Kindergesichter \emph{Hello} zugerufen. Das ländliche Leben Kambodschas inmitten der riesigen Reisfelder so zu erleben 
war wirklich ein eindringliches Erlebnis. Als wir wieder in Kampong Thom angekommen waren, fing es schon langsam an zu dämmern. 
Im Lonely Planet haben wir gelesen, dass es ein Spektakel der anderen Art neben einer alten französischen Residenz am Fluß zu sehen 
gibt. Hunderte Fledermäuse mit Spannweiten von über 40 Zentimetern hängen dort an drei Mahagonibäumen. Und bei Beginn der Dämmerung 
fangen diese an aufzuwachen. Wir haben uns also dort auf eine Bank gesetzt und gewartet bis die Fledermäuse sich auf die Jagd nach 
ihrem Futter gemacht haben. Ein schier endlos wirkender Strom an Fledermäusen der von den Bäumen über unsere Köpfe hinweg flog hat 
somit unseren perfekten Tag beendet.

23. November 2018, Self Service Laundry, Siem Reap

Angkor Wat! Der größte Tempel der Welt und für viele, wenn nicht alle, das Highlight von Angkor. Für uns beide war Angkor Wat 
auch sehr beeindruckend, aber nach ganz oben auf unsere Liste hat er es nicht gesschafft. Die ganze Tempelanlage von Angkor ist 
riesig. Man kann sich das erst vorstellen, wenn man dort von Tempel zu Tempel fährt. Es gibt mehrere Möglichkeiten, Tuk Tuk, Moped 
oder Fahrrad. Wir haben uns natürlich für das Fahrrad entschieden und so innerhalb von 3 Tagen mehr als 120 Kilometer abgeradelt.
Die beiden besten Tempel für mich waren Ta Prohm und Preah Khan. In beiden sind riesige Bäume zu sehen, die durch die Tempelgebäude 
wachsen und mit ihren massiven Wurzeln den Tempel langsam aber stetig wieder zurück erobern. Etwas schade war bei Ta Prohm, dass es 
viele Gerüste für Restaurierungsarbeiten zu sehen gab. So ist die Gesamtatmosphäre etwas beeinträchtigt, daher denke ich, dass mir 
Preah Khan am besten gefallen hat, auch wenn die Bäume und Wurzeln nicht ganz so massiv und beindruckend sind wie in Ta Prohm. Ein 
weiterer grandioser Tempel ist Bayon, der Tempel der 216 Gesichter. An quasi jeder Wand ist ein großes in Stein gehauenes Gesicht mit 
dem typischen Angkor-Lächeln zu sehen. Weiterhin gibt es sehr große Reliefs die auf der untersten Ebene den Tempel umrunden und die 
Geschichte des Königs und der Kriege gegen die Vietnamesen zeigen. Solche Reliefs sind auch in Angkor Wat zu sehen und nicht weniger 
grandios wie diejenigen in Bayon. Auch wenn Angkor Wat nicht das Highlight unter den Angkor Tempeln war ist der größte Tempel der 
Welt doch ein Muss für jede Kambodschareise und zweifelsohne eine der größten Errungenschaften menschlicher Schaffenskraft. Aber es 
ist auch bezeichnend, dass das schönste an Angkor Wat wohl der Sonnenaufgang mit dem Tempel als Kulisse und nicht der Tempel an sich.
Nach 3 Tagen die immer gegen 4:30 AM begonnen haben sind wir beide aber auch komplett platt und haben erst einmal genug von Tempel.
Es ist schon ermüdend, vor allen Dingen weil es jeden Tag über 30 Grad Celsius hatte und man ununterbrochen schwitzt. Glücklicherweise 
kann man überall erfrischende Fruit Shakes für 1 USD kaufen. Das ist aber auch wirklich notwendig wenn man einigermaßen frisch bleiben 
will. Heute ist unser letzter voller Tag in Kambodscha und wir nutzen die Zeit etwas zu entspannen, Postkarten zu schreiben, die 
Wäsche zu machen und noch ein paar gute Fruit Shakes zu genießen. Morgen in der Früh geht es dann in das vierte Land unserer Weltreise, 
Thailand.

Das war also Kambodscha, eines der ärmsten Länder der Welt! Der nur 40 Jahre zurück liegende Genozid der Roten Khmer ist an der 
Einstellung der Menschen nicht zu erkennen. Alle sind froh, nett und zuvor kommend. Die Kinder lachen einen noch mehr an als 
beispielsweise in Vietnam. Aber sobald man in ein Gespräch mit Einheimischen kommt merkt man doch, dass die Schreckenszeit Ende der 
der 70er Jahre noch sehr präsent in den Köpfen der Menschen ist. Aber das ist auch kein Wunder, da wahrscheinlich jeder Kambodschaner 
Mitglieder ihrer Familie durch die Roten Khmer verloren hat. Ansonsten ist auffällig, dass alles recht teuer ist. Ich weiß nicht ob das 
alles Touristenpreise sind, aber Essen und auch Hotels sind deutlich teurer in Vietnam, obwohl Kambodscha viel ärmer und weniger 
entwickelt ist. Das Essen hat mich auch nicht wirklich überzeugt, es fehlt meiner Meinung nach ein bisschen die Kreativität und 
Vielfalt. Aber ich denke das Essen von jedem Land hätte mich enttäuscht, nachdem ich 4 Wochen lang die vietnamesische Küche genießen 
konnte. Trotzdem ist Kambodscha auf jeden Fall eine Reise wert und ich bin mir sicher, dass die Tempel von Angkor eines der Highlights 
unserer ganzen Reise sein werden.


\chapter{Thailand}

28. November 2018, Zug Nummer 7 von Ayutthaya nach Chiang Mai

Vor 4 Tagen sind wir mit dem Bus von Siem Reap nach Bangkok gefahren. An der Grenze lief alles ohne Probleme. Obwohl Blandine 
so extrem nervös war, dass wir nicht einreisen  dürften, da wir keine Ausreisebestätigung ausgedruckt hatten. Alle diese Sorgen 
waren natürlich nicht notwendig und wir wurden quasi einfach durch die Grenze gewunken. Mit dem Bus hat auch alles super funktioniert.
Ich hatte im Vorfeld einen Horrorbericht über diese Busfirma gelesen, aber zumindest bei uns gab es nichts zu beanstanden. In Bangkok 
angekommen haben wir uns dann zu Fuß aufgemacht in die Backpacker-Straße \emph{Khao San}. Dort sind all die Bars und unser Hotel 
war ganz in der Nähe. Wir beide waren recht erschöpft (die drei Tage in Angkor waren wirklich anstrengend) und haben uns nach dem 
Check-In nur kurz was zu Essen geholt und sind dann in unsere Betten gefallen. 
Am nächsten Tag wollten wir zu erst den Royal Palace besichtigen. Auf dem Weg dorthin hat uns ein Tuk Tuk Fahrer aber gesagt, dass 
dieser den ganzen Vormittag geschlossen wäre. Später haben wir gelernt, dass das eine typische Betrügermasche der Fahrer ist, die 
den Touristen eine teure Bootstour als Ersatz vorschlagen. Wir haben unseren Plan aber davon beeinflussen lassen und sind erst 
in das National Museum gegangen. Danach sind wir zum Royal Palace gegangen, aber rein kamen wir trotzdem nicht, Leggings sind nicht 
erlaubt. So mussten wir wieder unseren Plan ändern und sind aus dem Zentrum rausgefahren um in einer Mall ein paar Sachen einzukaufen.
Am Tag darauf konnten wir dann endlich den Royal Palace besichtigen. Anfangs wirkt es schon sehr beeindruckend mit reicht verzierten 
Tempeln und Gebäuden. Alles ist golden und glitzert. Aber schon nach kurzer Zeit ermüdet es etwas das Auge, was durch die enge Anordnung 
der Gebäude noch verstärkt wird. Die große Anzahl an Touristen die sich durch die engen Wege quetscht tut ihr übriges. So ist für 
mich der Royal Palace kein Highlight der bisherigen Reise aber trotzdem ein Must-See bei einem Bangkokbesuch. Abends wollten wir
wir dann ein bisschen die Atmosphäre der Khao San erfahren und uns ein paar Drinks in einer Bar genehmigen. Beim Abendessen kamen 
wir in ein Gespräch mit einem netten älteren Ehepaar aus Dublin mit denen wir uns dann auch in eine recht schicke Bar gesetzt haben.
WIr haben uns sehr nett unterhalten, beide waren von unserem Vorhaben der Weltreise begeistert. Die Drinks waren recht teuer, aber 
ohne dass wir etwas machen konnten haben beide unsere Drinks bezahlt. So hatten wir einen schönen Abend ohne viel Geld ausgegeben zu haben.
Lange haben wir es aber auch nicht in der Khao San ausgehalten, einfach aufgrund der unfassbaren Lautstärke der Musik in den Bars. 
Gerade für Blandine ist das etwas viel.
Am nächsten Morgen mussten wir dann früh aufbrechen um mit dem Zug nach Ayutthaya zu fahren. Dort haben wir eine Nacht verbracht. 
Wir haben uns verschiedene Tempel in der Stadt angeschaut und die waren das bisherige Highlight Thailands für mich. Zwar größtenteils 
zerstört und verfallen haben diese Tempel einen besonderen Charme. Berühmt ist beispielsweise der Buddhakopf, der von den Wurzeln 
eines Bodhibaumes umringt ist. So hat mir der Tag in Ayutthaya sehr gut gefallen, bis ein Hund auf die Idee kam mir in den Knöchel zu 
beißen. Wir waren gerade auf den Weg zu einem liegenden Buddha als ein Hund von hinten an mich heran gelaufen ist und mir in den 
Knöchel gebissen hat. Glücklicherweise nicht fest, aber etwas geschockt war ich schon, dass ich laut aufschrie als ich den Hund 
plötzlich bemerkt hatte. Ich denke ich hatte Glück im Unglück. Es gibt nur zwei kleine Kratzer und keinen tiefen Biss. Ich werde 
also diese Kratzer gründlich reinigen und das beste hoffen. 

1. Dezember 2018, The Royal Guesthouse, Chiang Mai

Planmäßig wären es zehn Stunden Zugfahrt von Ayutthaya nach Chiang Mai gewesen. Am Ende waren es dank einer technischen Panne am 
Zug ungefähr 12 Stunden. Das klingt nervig, aber eigentlich war es ziemlich cool. Kurz vor dem Sonnenuntergang sind wir mitten im 
nirgendwo stehen geblieben und da wir gerade in Richtung Westen gefahren sind, hatten wir ein sehr schönen Bild vor uns. Schnurgerade
führten die Gleise weiter und die Abendsonne strahlte uns in das Gesicht. Es gab so eine schöne Fotomöglichkeit mit dem Zug. Was 
genau die Panne am Zug war konnten wir nicht rausfinden, aber das Hammer war das Werkzeug der Wahl und zwar an mehreren Stellen
des Zuges. Auch wurden wild Sicherungen ausgetauscht, Knöpfe gedrückt und Hebel bewegt. Am Ende hat ein neuer Zugwagon angedockt 
und uns nach Chiang Mai gezogen. 
Chiang Mai ist die zweitgrößte Stadt in Thailand und deutlich entspannter als Bangkok. Geprägt ist die Stimmung durch die vielen 
Nachtmärkte. Gestern haben wir eine Tagestour zu einem \emph{elephant sanctuary} gemacht. Dort konnten wir Essen für die Elefanten 
vorbereiten und sie dann per Hand füttern. Dabei ist man natürlich auch auf Tuchfühlung mit diesen großen Tieren gegangen und das 
war ein wirklich schönes Erlebnis. Es gab insgesamt vier Elefanten, von denen einer erst drei Jahre alt war. Trotzdem war dieser 
Babyelefant schon sehr groß und kräftig. Nach dem Füttern sind wir mit den Elefanten zu einem nahegelegenen Fluss gegangen und haben 
dort mit ihnen gebadet. Auch das war ziemlich cool wenn auch letztendlich nichts spektakuläres. Im Vorfeld hatten wir einige Reviews 
gelesen dass auch in den sanctuaries die Elefanten mit spitzen Gegenständen dazu bewegt werden dass zu machen was auf dem Programm 
steht. Blandine und ich haben versucht zu sehen ob das auch bei uns der Fall war. Die Mitarbeiter haben sich unserer Meinung schon 
etwas komisch und auffällig verhalten aber letztendlich konnten wir nicht wirklich sehen, ob sie die Elefanten nur mit ihrer Hand 
gedrückt haben oder mit spitzen Gegenständen nachgeholfen haben. So oder so sind diese Elefanten nicht so frei wie man das eigentlich 
gerne möchte. Man hat schon gemerkt, dass es ein klaren Programm gab und dass die Elefanten daran gewohnt waren. Beispielsweise 
gab es ein Schlammbad für die Elefanten. Während alle Touristen da waren haben sich die Elefanten fleißig mit Schlamm bespritzt, so dass 
es den Anschein hatte, sie genießen den kühlen Schlamm. Aber kaum sind wir wieder weg gegangen haben sie wie auf einen Schlag 
aufgehört. Daran erkennt man, dass sie sich wirklich nicht so frei bewegen wie es beworben wird. Trotzdem war es eine sehr schöne 
Erfahrung für uns. 

7. Dezember 2018, Sunset Beach Bungalow, Koh Jum 

Seit gestern sind wir auf einer kleinen ruhigen Insel namens Koh Jum. Diese Insel ist eine der weniger touristisch erschlossenen 
und das ist genau richtig für uns. Wir haben ein Bungalow direkt am Meer und das ist wirklich sehr idyllisch hier. Die erste Nacht 
war allerdings etwas anstrengend für Blandine. Als sie in das Bett gegangen ist hat sie einen Frosch entdeckt, der es sich auf 
unserem Moskitonetz gemütlich gemach hat. Dieser musste natürlich erst einmal von mir entfernt werden. Das hat sich als gar nicht 
so einfach heraus gestellt. Während dieser Aktion hat der Frosch auch aus Angst auf meine Seite des Bettes gepinkelt. Aber naja, 
so etwas muss man erdulden wenn man diese schöne Atmosphäre haben möchte. Der Tag heute war sehr entspannt. Morgens haben wir 
geschnorchelt. Das war das erste Mal für Blandine und nach anfänglichen Schwierigkeiten hat sie viel Freude daran gefunden. Es 
gab leider nicht mega viel zu sehen, aber einige größere und bunte Fische konnte man schon entdecken. Ich schaffe es hier auch wieder 
mehr mein Kung Fu zu trainieren, zwei mal pro Tag werde ich ohne Probleme schaffen. Morgen ist unser letzter voller Tag auf Koh Jum 
und am Tag darauf geht es dann zurück nach Krabi. 
Unsere letzten Tage in Chiang Mai waren auch ganz schön. Den Tag nach den Elefanten haben wir uns einen Ruhetag gegönnt und 
nicht sehr viel gemacht außer gelesen und uns entspannt. Abends habe ich mir einen Muay Thai Kampf angeschaut. Definitiv ein 
Teil der Thailand-Erfahrung. Insgesamt gab es 6 Kämpfe. Der erste Kampf waren zwei kleine Jungs, vielleicht 10 Jahre alt. Das 
war etwas bizarr, vor allen weil sie keinen Kopfschutz hatten und trotzdem ziemlich auf einander los gegangen sind. Dann gab es 
noch 2 Frauenkämpfe und 3 Kämpfe mit Männern. Alles in allem war es schon sehr interessant anzuschauen. Cooler wäre es gewesen, 
Kämpfe zu sehen wo nur Locals vor Ort sind und auf den Sieger wetten. So etwas hat es früher in Chiang Mai gegeben, aber mittlerweile 
sind eigentlich nur Touristen bei den Kämpfen vorzufinden. 
Am darauffolgenden Tag haben wir uns ein Moped ausgeliehen und sind nördlich von Chiang Mai herum gefahren. Das war für uns beide 
der schönste Tag in Chiang Mai. Wir sind zuerst zu dem \emph{Sticky Waterfall} gefahren. Das war schon ziemlich cool. Es ist nicht 
wirklich ein Wasserfall, sondern ein Bach der relativ steil hinab fließt. Es wurden mehrere Seile entlang des Baches befestigt, 
so dass man im Bach hinunter gehen konnte. Eine rutschige, aber sehr spaßige Angelegenheit. Nach dem Sticky Waterfall sind wir 
weiter zu einem richtigen Wasserfall gefahren. Von dort aus sind wir dann mehrere Stunden durch einen Nationalpark gefahren. So etwas
lieben wir beide. Wieder gab es mehrere Reisfelder zu sehen. Es wurden auch mehrere Felder abgebrannt und die Asche wurde verteilt. 
Am Ende des Tages gab es dann den wichtigsten Tempel Nordthailands zu besichtigen. Mittlerweile sind wir aber so von Tempeln gesättigt
dass und das wirklich überhaupt nicht beeindruckt hat. Am Abreisetag hatten wir noch sehr viel Zeit und haben das genutzt um uns mal 
richtig durch massieren zu lassen. Das hat uns beiden sehr gut getan und eigentlich haben wir vor, noch eine Ölmassage mitzunehmen 
bevor wir nach Malaysia reisen. 
Die Reise nach Krabi war dann relativ anstrengend. Wir sind geflogen, mit einem lay-over in Bangkok. Dort mussten wir 5 Stunden 
auf unseren nächsten FLug warten, mitten in der Nacht. Blandine hat versucht am Flughafen zu schlafen, aber der Schlaf war nicht 
wirklich erholsam. Aber trotz allem sind wir gut und vollständig in Krabi angekommen. In Krabi haben wir uns dann ein Moped ausgeliehen 
und sind zu einem Nature Trail gefahren. Der Trail war sehr schön und wir haben bestimmt 4 Stunden damit verbracht auf einen kleinen 
Hügel zu wandern. Aber dank der Hitze und der hohen Luftfeuchtigkeit kamen wir beide kräftig ins Schwitzen. Ganz oben gab es dann 
eine wundervolle Aussicht und eine schöne Fotomöglichkeit auf einem Felsvorsprung. Abends haben wir dann einen altern Bekannten wieder 
getroffen, Marc aus Spanien. Ihn und seine Freundin hatten wir in Emei, China kennen gelernt. Er ist ebenfalls auf Weltreise und es 
war super sich hier wieder getroffen zu haben. Wir waren zufällig am gleichen Ort aber das Treffen war verabredet. Wir konnten 
ein paar Geschichten austauschen, was uns während der letzten 2 Monate alles passiert ist. Seine Reise wird allerdings etwas unterbrochen 
weil er über die Feiertage zurück nach Spanien fliegen wird.

9. Dezember 2018, Chan Cha Lay Hotel, Krabi

Die letzte Nacht in Thailand. Morgen machen wir uns auf den Weg nach Malaysia. Wir haben auch die restlichen Tage auf Koh Jum 
sehr genießen können. Aber die Probleme mit unserem Bungalow haben leider nicht nach gelassen. Wir hatten während der letzten Nacht 
hunderte Ameisen auf unserem Moskitonetz und heute morgen habe ich etwas auf meiner Hose gesehen. Ich weiß nicht genau was es war, aber
es sah aus wie Dämmmaterial von dem Dach. Jedenfalls hat es unzählige Ameisen angezogen und meine ganze Hose war begraben von einem 
krabbelnden Haufen Ameisen. Naja, nicht so schön und das nächste Mal gehen wir wahrscheinlich eine Preisklasse höher. Gerade Blandine 
konnte sich die letzte Nacht gar nicht mehr entspannen. Für sie ist das wirklich schwierig. Trotzdem war die Inselatmosphäre sehr schön 
und für mich das Highlight der Thailandreise. 
Den halben Tag in Krabi haben wir dann hauptsächlich dazu genutzt nichts zu machen. Zumindest für mich, Blandine hat fleißig an 
ihren Impressionen geschrieben. Und wir haben uns noch eine schöne Ölmassage gegönnt. Zum Abendessen sind wir zum Nachtmarkt gegangen, 
wo es eine Vielzahl an Essensständen gab. Dort haben wir uns wirklich satt gegessen. Also alles in allem war es heute ein schöner 
Abschlusstag für unsere Thailandreise.


Thailand, das vierte Land unserer Weltreise. Obwohl es ein schönes Land zum Bereisen ist, muss ich sagen, dass ich hier 
die größte Enttäuschung verspürt habe. Das lag glaube ich nicht so sehr an hohen Erwartungen sondern zum einen dass das Land 
kulturell nicht so viel zu bieten hat und auch dass wir nach Angkor mit thailändischen Tempeln nun wirklich gar nichts mehr anfangen 
konnten. Trotzdem überwiegen natürlich die schönen Erfahrungen. Allen voran Koh Jum, die wunderschöne ruhige Insel die quasi unseren 
Abschluss der Thailandreise markiert hat. Auch in guter Erinnerung bleibt die Zugfahrt von Ayutthaya nach Chiang Mai, der Muay Thai 
Kampf und trotz leichter Bedenken die Bekanntschaft mit den Elefanten. Man muss zugeben, dass Thailand es nicht einfach hatte uns 
zu beeindrucken, nachdem wir schon 3 asiatische Länder bereist haben. Die Märkte sind nicht mehr beeindruckend, die Tempel verblassen 
durch ihre Eintönigkeit und das Essen bieten nichts Unentdecktes. Trotzdem hat es uns viel Freude bereitet dieses Land zu bereisen.



\chapter{Malaysia}

14. Dezember 2018, Bed Station Guesthouse, Tanah Rata

Die ersten Tage liegen in Malaysia liegen bereits hinter uns und bisher hat uns dieses Land nicht enttäuscht. Aber bis wir in 
George Town angekommen sind mussten wir einige Kilometer zurück legen. Von Krabi in Thailand haben wir einen Bus nach Hat Yai 
genommen. Hat Yai ist die größte Stadt in im Süden Thailands und von dort aus kommt man am einfachsten zur Grenze. In Hat Yai 
angekommen haben wir dann einen Mini Van nach Padang Besar genommen, der Grenzstadt nach Malaysia. Der Grenzübergang war dann 
sehr unkompliziert und weiter ging es dann mit einem Zug. Der Zug war eher eine S-Bahn, aber dafür war es sehr günstig. Der Zug 
hat uns aber auch noch nicht bis nach George Town gebracht, sondern nur nach Butterworth. George Town liegt auf einer Insel und so 
mussten wir noch mit einer Fähre rüber fahren. Insgesamt hat die Reise von Krabi nach George Town 14 Stunden gedauert und wir waren 
in 4 verschiedenen Verkehrsmitteln. Dem entsprechend waren wir relativ müde als wir endlich in unserem Hotel angekommen waren. 
Aber von da an waren wir von George Town einfach nur begeistert. Es gibt einen bunten Mix verschiedener Kulturen, da Inder, Muslime 
und Chinesen schon seit vielen Generationen dort leben. Daher gibt es mehrere China Towns, muslimische Viertel und auch ein little 
India. Vor 250 Jahren kamen dann auch die Briten auf die Insel und deren kolonialistischer Baustil ist nun auch ein großer Teil 
der Atmosphäre in George Town. Es gibt mehrere hundert \emph{heritage houses} und an jeder Ecke findet man Street Art. Es wird 
also wirklich nicht langweilig durch die Straßen zu schlendern. Eine willkommene Abwechslung war auch das Essen. Seitdem wir Vietnam 
verlassen haben gab es sehr häufig \emph{fried rice}. Ich habe besonders Naan, das indische Brot, und das Chicken Tandoori genossen.
Nachdem wir den ersten Tag damit verbracht haben die Stadt und besonders die Street Art zu erkunden haben wir uns für den zweiten 
Tag vorgenommen die Natur der Insel Penang etwas zu erkunden. Vormittags sind wir zu dem Penang National Park im Nord-Westen der 
Insel gefahren. Dort gab es leider nicht so viel zu tun, weil viele Wanderpfade gesperrt wurden. So konnten wir nur eine kleine 
Wanderung am Vormittag machen und dann hatten wir also schon alles gesehen. Was konnten wir also noch mit dem Nachmittag anfangen?
Es gibt noch den Penang Hill, mit 820 Metern der höchste Punkt Penangs. Die meisten Leute neben eine Bahn um nach oben zu fahren, 
aber das ist natürlich nichts für uns. Wir haben uns für einen Pfad entschieden, der über 12 Kilometer durch dichten Dschungel führt.
Anfangs wurden wir noch von einem Chinesen aus Singapur begleitet, den wir auf dem Weg kennen gelernt haben und der sich uns anschließen 
wollte. Aber schon nach einer halben Stunde hat er entkräftet aufgegeben. Es war aber auch sehr heiß und schwül und der Weg war 
relativ anstrengend, da es wirklich durch den Dschungel ging. Alle paar Meter musste man über umgefallene Bäume klettern, an Wurzeln 
hoch klettern oder sich durch hüfthohes Gewächs kämpfen. Nach einer Stunde wurde es dann auf einmal immer dunkler, weil große 
Wolken aufgezogen sind. In der Ferne gab es lauten Donner, der immer näher zu kommen schien. Uns wurde also deutlich gemacht, dass 
wir nicht all zu viel rum trödeln sollten. Es war schon 15h30 und der Sonnenuntergang war nur noch 3,5 Stunden entfernt. Eigentlich 
genügend Zeit für die Strecke aber wir hatten unterschätzt wie langsam wir uns doch durch den Dschungel bewegen würden. Mit der Zeit
hingen auch die Wolken immer tiefer und es fing an zu regnen. Es wurde also wirklich eine Frage des Willens sich möglichst schnell 
durch den Dschungel zu kämpfen, damit wir uns nicht in der Dunkelheit verlieren würden. Am Ende haben wir es aber gerade noch rechtzeitig 
geschafft, wenn auch unter großer Anstrengung und völlig durchnässt, durch Schweiß und Regen. Aber es war eine super Erfahrung 
und genau dafür machen wir ja auch diese Weltreise. Den letzten Tag in George Town sind wir dann sehr ruhig angegangen, weil wir 
uns wirklich entspannen mussten. Wir haben uns aber eine der Hauptsehenswürdigkeiten George Towns angeschaut, das Blue Mansion. 
Das oppulente Wohnhaus von Cheong Fatt Tze, dem reichsten Chinesen seiner Zeit. Es gab einen Guide der viel zur Geschichte des Hauses 
und von Cheong Fatt Tze erzählt hat und so war es wirklich ein interessanter Besuch.
Heute sind wir dann in die Cameron Highlands gefahren. Hier wird ein Großteil des Tees in Malaysia angepflanzt. Die Fahrt hat ungefähr 
8 Stunden gedauert weil wir besondern die letzten 20 Kilometer nur im Schritttempo vorwärts kamen. Landschaftlich war die Fahrt 
aber sehr schön. Fast die ganze Fahrt über gab es Dschungel neben der Autobahn und ich habe so etwas noch nie wirklich gesehen und 
so wurde mir auch nich langweilig aus dem Fenster zu schauen. Cameron Highlands sind ungefähr 1600m über dem Meeresspiegel und so 
ging es am Ende der Busfahrt auch ordentlich bergauf. Hier war es besonders spektakulär zu sehen wir tief die Wolken hingen, sodass 
die Gipfel der Berge komplett in ihnen eingehüllt waren. Auch wir sind auf einmal mitten in einer Wolke gewesen. Das war schon 
ziemlich cool.

16. Dezember 2018, Hana Guesthouse, Kuala Tahan

Wir haben einen Tag in Tanah Rata verbracht und den auch nicht so vollgepackt wie üblich. Unser Plan war es eigentlich eine Teeplantage 
zu besuchen und auch den \emph{Mossy Forest} zu besichtigen. Allerdings war es nur möglich den Mossy Forest mit einer gebuchten 
Tour zu besuchen und darauf hatten wir wirklich keine Lust. Zur Zeit sind Ferien in Malaysia und an Wochenenden sind die Cameron 
Highlands auch immer sehr gut besucht. So hatten wir Sorge, dass die Tour überhaupt keinen Spass machen würde und wir auch viel Zeit 
im Stau stehen würden. So haben wir einfach ein Moped für einige Stunde gemietet und sind zu einer Teeplantage gefahren die nicht 
so touristisch sein soll. Das war auch eine sehr gute Entscheidung. Wir waren quasi alleine auf der Plantage und konnten die Aussicht 
genießen. Es gab auch eine kleine Besichtung der Teeverarbeitungsanlage. So war es ein recht interessanter Besuch. Danach sind wir wieder 
in die Stadt gefahren und haben den restlichen Tag damit verbracht Postkarten zu schreiben und uns zu entspannen. Es gab auch endlich 
mal wieder Zeit für ein kleines Mittagsschläfchen. 
Heute haben wir uns dann wieder auf den Weg gemacht zu dem nächsten Ort, Tamana Negara, den ältesten Regenwald der Welt, so wird 
es zumindest behauptet. Wir sind mit dem Minivan ein paar Stunden gefahren und die letzten 40 Kilometer ging es dann Flussaufwärts 
mitten in den Dschungel. Leider sind aufgrund der Monsunzeit einige Sachen gesperrt und so können wir nicht alles machen, was wir 
uns vorgenommen hatten, aber sicherlich werden wir einige Stunden schwitzend durch den Nationalpark wandern.

18. Dezember 2018, Hana Guesthouse, Kuala Tahan

Der Dschungel ist schon anstrengend. Gestern sind wir in der Früh mit einem Boot zum anderen Flussufer gefahren um zu einem Wasserfall 
zu trekken. Es hat die ganze Nacht über geregnet und auch als wir los gegangen sind hat es noch leicht genieselt. Daher war der ganze 
Boden komplett matschig und es dauerte nur einige Minuten bis unsere Schuhe komplett von Schlamm bedeckt waren. Die Wanderschuhe 
haben sich also schon definitiv als nützlich erwiesen. Der Dschungel hier im Taman Negara ist auf jeden Fall wilder als der, den
wir auf Penang gesehen haben. Aber die Wege sind hier besser gepflegt und nicht so zu gewachsen. Auch war es eine sehr flache Route 
und daher bei weitem nicht so anstrenged wie der Penang Hill. Allerdings gab es überall Blutegel, die sich über unsere Schuhe und Hose 
immer weiter nach oben bewegt haben. Besonders für Blandine war das schwierig. Sie liebt zwar die Natur aber diese Tiere lassen 
sie fast schon panisch werden. So war sie immer wieder damit beschäftigt die Blutegel von ihrer Hose zu entfernen. Das ist viel 
schwieriger als es klingt, denn sie sind sehr glitschig und können sich sehr an der Hose fest saugen. Relativ häufig konnte ich also 
sehen wie Blandine versuchte mit den Fingern den Blutegel zu packen. Wenn das aber gelungen ist, hat er sich schon an der Hand 
festgesaugt und Blandine hat panisch auf ihre Hand eingeschlagen. Das war wirklich nicht einfach für sie, aber sie hat es am Ende 
ohne Blutegel am Körper überstanden. Ich habe nicht so sehr darauf geachtet und daher konnten sich so einige Blutegel an meinem 
Blut satt fressen. Der Trek an sich war schön aber auch nichts besonderes. Der Dschungel ist schon interessant aber nach einer Weile 
geht man einfach durch ohne der Natur große Beachtung zu schenken. Aber natürlich gib es so einige coole Sachen zu sehen, wie rieseige 
Bäume die wohl aufgrund von Altersschwäche gestorben und umgefallen sind und es gab auch den größten Baum in Malaysia zu sehen, der 
wie schon zu vermuten, wirklich riesig ist. Fast an den Wasserfällen angekommen wurden wir dann aber mächtig enttäuscht. Man hätte 
einen Fluss überqueren müssen und wir hatten erwartet dass es dort eine Brücke oder ähnliches geben würde. Gab es aber nicht, 
nur einen Fluss der nicht zu überqueren war. Wahrscheinlich ist er nur während der Trockenzeit, wenn es nicht so viel Wasser gibt, 
per Fuß zu überqueren. So sind wir wieder umgekehrt und haben uns relativ zügig auf den Weg zurück zum Hotel gemacht. Auf dem 
Rückweg haben wir dann auf einmal aggressive Laute von einem Elefanten gehört. Es klang relativ weit entfernt aber war doch 
sehr einschüchtend. Auf dem Hinweg haben wir schon sehr viele Elefantenhaufen gesehen. Blandine hatte noch etwas mehr Respekt 
vor dem Dschungel bekommen und wollte jetzt wirklich schnell aus dem Dschungel raus. Auf dem Rückweg haben wir uns dann entschlossen 
heute einen Ruhetag einzulegen. ursprünglich wollten wir heute wieder in den Dschungel aber so richtig Lust hatten wir beiden nicht, 
da es im Endeffekt der gleich Tag werden würde wie davor. Am Abend haben wir dann 2 Schweden wieder getroffen die den gleichen Trek 
wie wir gemacht haben. Sie hatten weiter flussaufwärts eine Stelle zum überqueren gesucht aber auch nichts gefunden. Wir haben uns
dann für ein paar Bier verabredet und so noch einen schönen Abend verbracht. 

21. Dezember 2018, Dorms KL 2, Kuala Lumpur

Unseren letzten Tag in Taman Negara sind wir sehr ruhig angegangen, aber trotzdem gab es abends das Highlight unseren Aufenthaltes. 
Wir haben einen Night Safari gebucht, einen 90 minütigen Walk durch den Dschungel nach Sonnenuntergang mit einem Guide. Das war 
wirklich nochmal eine andere Art den Dschungel zu erfahren. Sehr viele Tiere, hauptsächlich Insekten, Spinnen und Frösche sind nachtaktiv 
und so konnten wir mit Hilfe des Guides sehr sehr viele Tiere sehen. Da gab es große Frösche, deren Urin zu Blindheit führen kann, 
wenn man ihn in das Auge bekommt, Spinnen die an Bäumen verharren aber jederzeit auf ihre Opfer oder uns Menschen springen können 
und viele Insekten wie Stabheuschrecken und giftige Hundertfüßler. Das erste größere Tier, welches wir gesehen haben war ein Trogon, ein
sehr seltener Vogel der ruhig auf einem Ast geschlafen hat. Das beeindruckendste Erlebnis kam dann aber als der Guide zu uns gesagt hat, 
dass wir alle unser Licht ausmachen und uns nicht bewegen sollten. Er hat dann nur eine Art bläuliches Licht angemacht und dann haben 
wir es gesehen, ein Skorpion kam aus seinem Loch heraus gekrochen, sehr gut zu erkennen weil er intensiv unter dem blauen Licht geleuchtet 
hat. Was wir nicht wussten, dass diese Art von Skorpionen auto-luminiszierend ist. Das war wirklich etwas, was ich so noch nie gesehen 
habe. Außerdem ist dieser Skorpion einer der 5 giftigsten Skorpionen auf diesem Planeten. Alleine dafür hat sich diese Night Safari also 
schon gelohnt. Direkt nach den Skorpionen haben wir dann noch eine grüne Schlange auf einem Ast "sitzend" gesehen. Diese Art von 
Schlangen ist nicht giftig, aber man möchte trotzdem nicht, dass sie sich von einem Ast abseilen und sich auf den eigenen Kopft setzen.
Das letzte Tier, welches ich dann noch sehen konnte war ein Hirsch. Diese Hirschart ist aber einer der kleinsten auf der Welt und 
war nur so groß wie ein Hase. Man konnte ihn nicht gut sehen, weil er etwas weiter entfernt im Gebüsch war, aber die Augen haben 
im Licht der Taschenlampe stark geleuchtet. Das war also unsere Night Safari und die hat sich wirklich gelohnt. So haben wir den 
Dschungel noch einmal mehr zu schätzen gelernt und wir wissen jetzt, dass wir definitiv keine Nacht ohne Schutz um Dschungel verbringen 
wollen.
Am nächsten Tag sind wir dann nach Kuala Lumpur aufgebrochen. Dort angekommen sind wir in unser Hostel eingecheckt, haben eine Kleinigkeit 
gegessen und sind dann gleich zu den Petronas Towers gegangen. Als ich mein allererstes Guiness Buch der Weltrekorde bekommen habe, 
waren die Petronas Towers das höchste Gebäude der Welt und ich habe mich schon gefreut sie hier in Kuala Lumpur zu sehen. Sie wirken 
mittlerweile nicht mehr so hoch, besonders weil wir schon in Hong Kong, Shenzhen und Ho Chi Minh City höhere Gebäude gesehen haben, aber 
die Architektur lädt dazu ein viele Fotos zu nehmen. Gestern haben wir dann endlich mal wieder ein paar Museen besichtigt. Zuerst das 
Islamic Arts Museum und dann das National History Museum. Beide waren recht gut, auch wenn mir das National Museum besser gefallen hat.
Heute sind wir zu den Batu Caves gefahren. Dort gibt es einen Hindutempel zu besichtigen der absolut nichts besonders war. Aber es 
gab die sogenannte Dark Cave in der wir eine Educational Tour mit einem Guide gemacht haben. Das war sehr interessant und wir konnten 
ein paar Tiere sehen die in der Höhle leben und haben mehr über das Ökosystem gelernt. Am Nachmittag sind wir dann in der Stadt über 
das Museum der Illusionen gestolpert, welches wir dann besichtigt haben. Das war ziemlich cool, besonders weil es viele Rätselspiele 
gab an denen wir uns intellektuel messen konnten.

26. Dezember 2018, Kota Kinablu International Airport, Kota Kinabalu

Das sind wirklich außergewöhnliche Weihnachten die Blandine und ich dieses Jahr verbracht haben. Wir sind vor 3 Tagen nach Borneo 
geflogen, mit dem Ziel den Mount Kinablu, den höchsten Berg Südostasiens, zu besteigen. Um 6 Uhr morgens am 24. Dezemeber wurden 
wir von unserem Hostel abgeholt und zum dem Kinabalu Nationalpark gefahren. Am ersten Tag sind wir von Parkeingang auf ungefähr 
1800 Meter Höhe zu unserer Hütte auf 3200 Meter gewandert. Also direkt mit 1400 Höhenmeter einsteigen war schon nicht schlecht. Aber 
da das ganz am Anfang war, waren Blandine und ich noch beide sehr fit und wir haben diese insgesamt 6 Kilometer Strecke in 3 Stunden 
abgerissen. Unser Mountain Guide kam dabei auch ganz schön ins Schwitzen. Wahrscheinlich ist er es nicht gewohnt, dass die Leute 
so den Berg hoch rasen. Auf der Hütte angekommen hatten wir so noch den ganzen Nachmittag zum Entspannen und Ausruhen. Gegen 4 Uhr 
nachmittags gab es dann ein Briefing für den nächsten Tag, also der Weg zum Gipfel und die Via Ferata, das eigentliche Highlight des 
Kinabalus. Nachdem wir das Briefing beendet haben gab es dann unser Weihnachtsessen, ein Buffet mit verschiedensten Sachen. Das 
war ganz gut, weil wir uns so richtig satt essen konnten vor einem sehr anstrengenden ersten Weihnachtsfeiertag. Während des Abendessens 
gab es auch einen wunderschönen Sonnenuntergang zu sehen. Die Wolken hängen sehr tief auf Borneo und unter uns gab es eine geschlossene 
Wolkendecke die von einer feuerroten Sonne bestrahlt wurde. Das war wirklich ein ganz besonderer und wunderschöner Anblick. Nach dem 
Essen gab es dann aber auch nicht viel mehr zu machen als direkt ins Bett zu gehen. Am nächsten Tag hat dann der Wecker schon um 
1:45 in der Nacht geklingelt. Der Plan war nämlich um 2:30 von der Hütte aufzubrechen um den Gipfel vor dem Sonnenaufgang zu erreichen.
Dieser Aufstieg war auch etwas zauberhaftes. Es war fast Vollmond und da es auch keine Wolken gab konnte man auch ohne Lampe 
sehr gut sehen. Der ganze Berg war in dem weichen Licht des Mondes bestrahlt. Blandine und ich waren, wie üblich, auch die schnellsten und 
so gab es keine störenden Taschenlampen, da die anderen Bergsteiger deutlich hinter uns waren. Das war wirklich eine meiner schönsten 
Erfahrungen bisher das felsige Plateau vor dem Gipfel zu durchschreiten während das Mondlicht eine schöne mystische Atmosphäre erschafft.
Um 4:40 sind wir dann sehr überpunktlich auf dem Gipfel angekommen. Aber das war sehr gut so. Auf dem Weg zum Gipfel und die ersten Minuten 
auf dem Gipfel war es eine sternenklare Nacht über uns. Man konnte verschiedenste Sternenbilder sehen und die Venus war extrem hell 
und deutlich zu sehen. Doch kurz nach unserer Gipfelankunft haben dichte Wolken den ganzen Berg umschlossen. Wir waren mitten in der Wolke 
und konnten vielleicht 10 Meter weit schauen. Das Mondlicht, welches vorher den ganzen Berg beleuchtet hat war nur noch ein diffuses 
Licht welches kaum durch die Wolke kam. Wenig später ist die Wolke aber weiter gezogen und es war wieder sternenklar. Wir konnten 
dann in der Ferne auch 2 Gewitter ausmachen. Das war auch etwas ganz besonderes, weil wir auf beide Gewitter hinab geschaut haben. Das 
war das erste Mal, dass ich überhalb eines Gewitters war. So schön das auch war, haben wir natürlich gehofft, dass das Gewitter 
nicht in unsere Richtung zieht. Nach einer frostigen halben Stunde sind wir dann wieder runter vom Gipfel um uns einen ruhigeren Ort 
für den Sonnenaufgang zu suchen. Dieser war dann aber nicht so wirklich spektakulär, weil es relativ bewölkt war. Dann ging es aber zur 
höchsten Via Ferata der Welt! Mir war im Vorfeld schon bewusst, dass es eine ziemliche Herausforderung für mich werden würde entlang 
der Felswand zu klettern. Aber ich suche ja immer nach Möglichkeiten meine Höhenangst zu bewältigen. Also haben wir uns zum Checkpoint 
aufgemacht wo unser Guide auch schon auf uns gewartet hat. Genau als wie die Ausrüstung angelegt haben kam eine schöne Wolke entlang 
gezogen und wir waren innerhalb von Sekunden klitschnass. Glücklicherweise war das aber kein Grund die Via Ferata abzusagen. Wir haben also 
gewartet bis die Wolke vorüber gezogen ist und haben uns dann in die sehr steile Felswand begeben. Ich war erstaunt, da ich kaum Höhenangst 
verspürt habe und das obwohl unter mir nichts als Luft war. Aber meine Hände haben sich trotzdem schon sehr an die kleinen aber 
stabilen Halterungen geklammert. Nach einer Weile war ich schon recht routiniert und ich bin nach hinten lehnend die Felswand fast schon 
entlang gelaufen. Das war wirklich eine außergewöhnliche Erfahrung, über uns der strahlend blaue Himmel und unter uns eine Wolkendecke, 
die nur gelegentlich den Blick auf den Fuß des Mount Kinabalu frei gegeben hat. Aber da war immer noch in einem Berg waren konnte das 
Wetter jede Sekunde umschlagen und das tat es auch. Auf einmal waren wir wieder inmitten einer großen Wolke und damit mitten in dem Regen.
Mit dem peitschenden Wind wurde es an der Felswand dem entsprechend schon sehr unangenehm, weil auch alles sehr rutschig wurde. Das hielt 
dann aber auch nur ungefähr 20 Minuten an und dann hat der Regen aufgehört. Neben der Felswand mussten wir auch eine sehr wackelige Holzhängebrücke 
und eine so genannte tibetische Brücke, die nur aus 3 Drahtseilen besteht, überwinden. Bei diesen beiden Brücken habe ich die größte 
Höhenangst empfunden. Aber auch das konnte ich ohne größere Probleme überwinden. Für die ganze Via Ferata haben wir insgesamt 2:45 
Stunden gebraucht und waren damit deutlich schneller als der Durchschnitt. Wir können also stolz auf uns sein. Wir sind dann wieder 
zur Hütte gegangen und haben dort ein zweites Frühstück verspeist. Danach sind wir aber auch gleich den Berg wieder runter gegangen, 
oder besser gesagt runter gerast. Nach 2 Stunden waren wir unten und sind zu unserem Hostel gefahren. Wir hatten nur Lust uns zu entspannen.
Diese beiden Tagen waren nämlich nicht nur wunderschön und atemberaubend sondern auch sehr kräftezehrend.


Malaysia, unser letztes Flächenland in Asien, war wirklich eine positive Überraschung. Georgetown mit seiner Vielzahl an Street Art 
und dem bunten Mix verschiedener Ethnien ist einfach nur wundervoll. Die Cameron Highlands waren aufgrund der Monsunzeit, weswegen viele 
Trails gesperrt waren, eine kleine Enttäuschung aber die Teeplantagen waren trotzdem sehenswert. Der Taman Negara war dann wieder 
sehr toll, auch wenn wir seitdem vom Dschungel eher genug haben. Aber alleine schon für die Nacht Safari hat es sich gelohnt. Und das 
große Highlight Malaysias und auch der Weltreise war natürlich der Mount Kinabalu, ein grandioser Abschluss für Asien. Die Menschen 
waren auch sehr nett und hilfsbereit und es ist schön zu sehen wie verschiedenste Kulturen und Religionen nebeneinander voller Akzeptanz 
existieren können. Aufgrund der großen indischen Community konnten wir auch kulinarisch etwas Abwechslung erleben, nachdem wir monatelang 
von Fried Rice gelebt haben. Also alles in allem ein wirklich schönes Land, das eines meiner favorites in Südostasien ist.


\chapter{Singapur}

28. Dezember 2018, Footprints Hostel, Singapur

Singapur, unser siebtes und letztes Land in Asien, auch wenn es im Grunde nur eine Stadt ist. Im Vorfeld hat man von Singapur 
hautpsächlich gehört wie sauber und modern es ist und das stimmt auch uneingeschränkt. Singapur ist mit Abstand die sauberste und 
modernste Stadt, die wir bisher auf unserer Reise gesehen haben. So richtig viel erleben konnten wir bisher aber nicht. Das liegt 
zum einen daran, dass es, zumindest für uns, nicht so extrem viel zu sehen gibt, aber hauptsächlich waren wir einfach zu schlapp viel 
zu entdecken. Der Mount Kinabalu hat ziemlich an unseren Reserven genagt und gestern waren wir beide komplett schlapp und auch etwas 
krank. Wir haben quasi den ganzen Tag geschlafen und jeder Schritt hat geschmerzt. Am Tag zuvor, als wir in Singapur angekommen sind, 
haben wir noch etwas die Stadt erkundet. Wir sind entlang des Flusses gegangen und haben die Skyline und die alten Kolonialgebäude bewundert. 
Am Ende waren es dann wieder knapp 30000 Schritte an einem Nachmittag und das war vielleicht etwas viel. Glücklicherweise ging es uns 
heute wieder deutlich besser und wir konnten einen normalen Tag hinlegen. Wir sind vormittags zu dem S.E.A Aquarium gefahren. Für mich 
war es das erst Mal, dass ich in so einem großen Aquarium war und es war wirklich beeindruckend. Aber gerade bei den Haien hatte ich 
wieder das Gefühl, dass es den Tieren dort einfach nicht wirklich gut geht. Ich denke also ich werde in Zukunft solche Aquarien meiden. 
Trotzdem war es schon faszinierend verschiedenste riesige oder farbenfrohe Fische zu sehen. Die erhabenen Rochen waren für mich 
das schöne in dem Aquarium. Danach haben wir uns langsam in Richtung \emph{Gardens by the bay} aufgemacht. Eine recht schöne 
Parkanlage mit riesigen artifical trees. Aber in Dokus oder auf Fotos wirken diese trotzdem beeindruckender. Danach waren wir aber 
auch wieder recht müde und haben uns auf dem Weg ins Hostel gemacht um noch etwas zu entspannen.



Top Natur:
Tiger Leaping Gorge
Ha Long Bay
Reisfelder bei Kampong Thom
Shilin Stone Forest
Koh Jum

Top Kultur:
Angkor Wat
Great Wall of China
Verbotene Stadt von Hue + alte Kaisergräber 
Tuol Sleng Torture Prison

Menschen:
Yao Hongyi, ein Chinese der mit dem Fahrrad durch Tibet gefahren ist und der seinen ersten Trek durch die Tigerschlucht so sehr 
genossen hat, dass er einen seiner Wanderstöcke dort gelassen hat.

Tom, ein Israeli der zu viel Gras in der Tigerschlucht gekauft hat so dass ich ihm kräftig helfen musste es los zu werden.

Macha, eine Russin die in Hamburg studiert und in China am liebsten in Restaurants gegangen ist in denen niemand gegessen hat, damit sich 
die Kellner mehr Zeit für sie nehmen können.

Die beiden chinesischen Saisonarbeiter in unserem Hostel in Dali, die den ganzen Tag Tee trinkend, rauchend und mit dem Handy spielend 
auf der Couch in der Lobby verbracht haben.

Marc, ein Spanier der ebenfalls auf Weltreise ist diese aber kurzeitig unterbrochen hat um für eine Jobmöglichkeit nach Spanien zu fliegen 
obwohl er den Job gar nicht haben will.

David, ein Brite der mit dem Zug von England bis nach Singapur fahren wollte um seine Tochter zu besuchen. Er war wirklich ein begeisterter
Fan von Zügen und ein oldschool traveler.

Guido, ein Schweizer der 5 Wochen durch China und Tibet reist ohne ein Wort Englisch zu sprechen oder mit Stäbchen essen zu können.

Anne-Laure, eine Französin die jedes Jahr 5 Monate reist, 5 Monate arbeitet und sich jeweils einen Monat Zeit nimmt um sich von beidem 
zu erholen. Wobei sie auch beim Reisen kein Problem hat 2 Wochen lang nichts zu unternehmen.



\end{document}