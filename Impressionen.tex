\documentclass[11pt]{book}
\usepackage[utf8]{inputenc}
\title{Impressionen}
\author{Alexander Böhm}
\begin{document}
\maketitle
\chapter{Vorbereitungen}
1. September 2019, 12:04 im Zug von München nach Paris, es ist grau und regnet leicht

Wir sitzen im Zug. Nach sehr anstrengenden letzten Tagen haben wir es geschafft mit voll gepacktem
Rucksack aufzubrechen. Kurz vor der Abreise wurde es, wie immer, stressig. Es gibt einfach zu vieles
Dinge die man zu erledigen hat, Wohnung leer räumen, sich von Freunden verabschieden, Arzttermine, 
Behördengänge, und bei all dem nicht den Kopf verlieren während man panisch versucht nichts zu 
vergessen. Ich hatte erwartet, dass sich die Erleichterung, das Gefühl der Entspannung, im Zug 
einstellen würden. Aber ich habe ab dem Zeitpunkt ab dem ich die Wohnung verlassen habe auch den 
Stress hinter mir gelassen. Im Moment empfinde ich nur sehr große Vorfreude, eine schöne Mischung
aus positiver Nervosität und Aufregung. Dies führt dazu, dass ich trotz der körperlichen Müdigkeit
nicht in der Lage zu schlafen. Blandine und ich sind beide froh, dass wir noch 2 Tage in
Paris verbringen können bevor wir wirklich nach Hongkong aufbrechen. Ich denke, dass so ein verzögerter Start doch 
einige Ruhe mit sich bringt, da man nicht direkt zum Flughafen hetzen muss. 

3. September 2018, 14:45 am Gate, die Sonne scheint

Wir sitzen am Gate C91 und der A380-800 wartet schon auf uns. Das Gepäck ist abgegeben, zwei gut 
gepackte Rucksäcke die vom Gewicht etwas ungleich verteilt sind. Meiner wiegt 17 kg während 
Blandines bei ungefähr 11kg eingecheckt wurde. Das wird im Laufe der Reise sicherlich noch etwas 
besser aufgeteilt werden, aber mein Rucksack wirkt jetzt schon fast leer. Wir haben wohl zu 
effizient gepackt. Blandine wirkt sehr entspannt und ruhig, auch wenn sie innerlich sehr aufgeregt 
ist. Ihre Vorfreude auf diese Reise ist ihr die ganze Zeit anzumerken. Ich freue mich so sehr mit 
ihr diese Erfahrung zu machen. 
\end{document}